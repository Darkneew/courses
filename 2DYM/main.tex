\documentclass[a4paper]{report}

% packages % 
\usepackage[utf8]{inputenc} 
\usepackage{fvextra}
\usepackage{csquotes}
\usepackage[french, italian, spanish, english]{babel}
\usepackage[T1]{fontenc}   
\usepackage{color}  
\usepackage{amsmath, dsfont, amssymb, amsthm, stmaryrd}
\usepackage[style=alphabetic]{biblatex}
\usepackage{enumitem}
\usepackage[hidelinks]{hyperref}

% graphics %
\usepackage{graphicx}
\graphicspath{ {./images/} }

% environments %
\newtheorem{theorem}{Theorem}[section]
\newtheorem{corollary}{Corollary}[theorem]
\newtheorem{lemma}[theorem]{Lemma}

\theoremstyle{definition}
\newtheorem{definition}{Definition}[section]

\theoremstyle{remark}
\newtheorem*{remark}{Remark}
\newtheorem*{example}{Example}



% bibliography %
\bibliography{bibliography} 

\begin{document}

% title %
\title{La mesure de Yang-Mills en 2 dimensions}
\author{Buisine Léo\\Ecole Normale Superieure of Paris}
\maketitle

\tableofcontents

\chapter{Introduction}
\section{Variete differentielle}
\begin{definition}
    Variete differentielle : variete topologique separe localement modelee sur $\mathbb{R}^n$ de maniere lisse, aka munit de cartes et de changements de cartes (un atlas).
\end{definition}
\begin{definition}
    Espace tangent $T_pM$, l'espace des tangeantes a $M$ en $p$. Si $M$ est lisse de dim $n$, alors $T_pM$ est homeomorphe a $\mathbb{R}^n$.
\end{definition}

\begin{example}
    Un groupe de Lie $G$ est un groupe munit d'une structure de variete differentielle compatible. $\mathfrak{g} = T_eG$ est l'algebre de Lie associee. 
\end{example}

\section{Fibre vectoriel}
\begin{definition}
    Un fibre vectoriel de base $M$ de rang $k$ est une variete $E$ munie d'une surjection lisse $\pi : E\rightarrow M$ qui ressemble localement a $M \times \mathbb{R}^k$. On note $E_x \equiv \pi^{-1}(x)$ la fibre au dessus de $x$. 
\end{definition}

\begin{example}
    Le fibre tangeant $TM = \bigcup T_p M$. Le fibre cotangeant $T^*M$ en prenant le dual. 
\end{example}

\begin{definition}
    Soit $E, \pi, M$ un fibre. Une section locale de $E$ est une application $\sigma: U \subset M \rightarrow E$ tq $\pi \circ \sigma = \text{id}$. On note $\Gamma(M)$ l'espace des sections de $M$. 
\end{definition}

\begin{example}
    Les champs de vecteurs sur $M$: $\Gamma(TM) = \chi(M)$. Les 1-formes differentielles sur $M$: $\Gamma(T^*M) = \Omega^1(M)$. Par exemple, si $X\in \chi(M)$, $X_p = \sum_{i=1}^n X_i(x)\frac{\partial}{\partial x^i}$ et si $\omega \in \Omega^1(M)$, $\omega_p = \sum_{i=1}^n \omega_i(p) \text{d}x_i$
\end{example}

\section{Connexion}

$\nabla$ est la generalisation des derivees directionelles $\nabla_X S$, avec $X \in \chi(M)$, $S \in \Gamma(E)$

\begin{definition}
    Une connexion sur $(E, \pi, M)$ est au choix 
    \begin{itemize}
        \item Une application bilineaire $\nabla: \chi(M) \times \Gamma(E) \rightarrow \Gamma(E)$ qui verifie 
        \begin{equation}
            \begin{aligned}
                &\nabla_{f.X}S = f\nabla_X S \\ 
                &\nabla_X (fS) = f\nabla_X s + (Xf).s 
            \end{aligned}
        \end{equation}
        \item Une application lineaire $\nabla : \Gamma(E) \rightarrow \Omega^1(M,E)$ definit par $S\rightarrow (X \rightarrow \nabla_X S)$ qui verifie la formule de Leibniz juste au dessus  
    \end{itemize}
\end{definition}
\begin{theorem}
    L'espace $\mathcal A$ des connexions sur $E$ est un espace affine de direction $\Omega^1 (M, \text{End}(E))$. En d'autres termes, si $\nabla$ est une connexion et $A \in \Omega^1 (M, \text{End}(E))$, aors $A + \nabla$ est une connexion.
\end{theorem}

\section{Fibre principaux} 
$G$ groupe de Lie. Il agit sur lui meme par conjugaison, et sur son algebre par le releve de la conjugaison. 
\begin{definition}
    Soit $G$ un groupe de Lie. Un $G$-fibre principal est une variete diff $P$ munie d'une action libre a droite $P\times G \rightarrow P$, $(p,g) \rightarrow p.g$ dont les fibres $\pi^{-1}(x)$, $x\in M$ sont les orbites de $P$ sous l'action de $G$. 
\end{definition}

\begin{definition}
    Section: comme avant
\end{definition}

\begin{definition}
    Soit $(P, \pi, M)$ un $G$-fibre principal. Une connexion sur $P$ est une application $p\in P \rightarrow H_p \subset T_p P$ qui verifie 
    \begin{enumerate}
        \item $T_p P = H_p \oplus V_p$ ou $V_p = \ker ((\text{d}\pi)_p)$
        \item $H_p$, $p \in P$ est stable par action de $G$. 
        \item $p \rightarrow H_p$ est lisse. 
    \end{enumerate}
\end{definition}

Soit $X\in \mathfrak g$. On lui associe le champs de vecteurs fondamental $\tilde X$ sur $P$ defini par 
\begin{equation}
    \tilde X_p = \frac{\text{d}}{\text{d}t}\Big|_{t=0} p. e^{tX}
\end{equation}

\begin{definition}
    2eme def. $(P, \pi, M)$. Une connexion sur $P$ est une 1-forme differentielle $\omega \in \Omega^1 (P, \mathfrak{g})$ tq 
    \begin{enumerate}
        \item $\omega(\tilde X_p) = X$ pour tout $p\in P, X \in \mathfrak{g}$
        \item $(R_g)^{-1} \omega = \text{Ad}(g^{-1})\omega$ pour tout $g\in G$. 
    \end{enumerate}
\end{definition}

\begin{theorem}
    Les deux definitions coincident. 
\end{theorem}

\begin{theorem}
    L'espace $\mathcal A$ des connexions sur $P$ est un espace affine de direction $\Omega^1 (M, \text{ad}(p))$ (c'est un fibre vectoriel de fibre $\mathfrak{g}$)
\end{theorem}

\begin{definition}
    Soit $\omega$ une connexion sur $P$. Sa courbure est la 2-forme $\Omega \in \Omega^2 (P, \mathfrak{g})$ definie par $\Omega = \text{d}\omega + \frac{1}{2}[\omega \wedge \omega]$ ou $[\omega \wedge \omega]$ est definie par $[\omega \wedge \omega](x,y) = 2[\omega(x), \omega(y)]$. $\omega$ est dite plate si $\Omega = 0$. 
\end{definition}

\section{La mesure de Yang-Mills}
La Theorie de Yang Mills est une theorie de jauge (des equa diff sur des sections de fibre) non-abelienne. La theorie de YM avec $G = SU(2)\times U(1) \times SU(3)$ est le modele standard ($U(1)$ pour la charge electrique, $SU(3)$ pour la couleur, et $SU(2)$ pour la charge faible). L'espace temps est la variete, l'espace des champs est le $G$-fibre principal.\par \medskip 

La mesure de Yang Mills euclidienne est une mesure sur $\mathcal A$ definie par 

\begin{equation}
    d\mu _{YM} (\omega) = \frac{1}{Z} e^{-\frac{1}{2T} S_{YM}(\omega)} \text{d}\omega
\end{equation}
Avec $Z$ la fonction de partition, $T$ la constante de couplage, $\text{d}\omega$ est une "mesure de Lebesgue" sur $\mathcal A$, et $S$ l'action de Yang Mills definie par  
\begin{equation}
    S_{YM}(\omega) = ||\Omega||^2 = \frac{1}{2} \int_M <\Omega \wedge *\Omega>
\end{equation}

Avec $M = \mathbb{R}^4$, la construction rigoureuse de cette mesure est un probleme du millenaire. Mais pour $M$ une surface, on a une definition correcte. C'est l'objet de ce cours. 
\chapter{Construction de la mesure}

\chapter{Calcul de la fonction de partition}

\chapter{Etude asymptotique}

\end{document}