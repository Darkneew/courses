\documentclass[a4paper]{book}

% packages % 
\usepackage[utf8]{inputenc} 
\usepackage{fvextra}
\usepackage{csquotes}
\usepackage[french, italian, spanish, english]{babel}
\usepackage[T1]{fontenc}   
\usepackage{color}  
\usepackage{amsmath, dsfont, amssymb, amsthm, stmaryrd}
\usepackage[style=alphabetic]{biblatex}
\usepackage{enumitem}
\usepackage[hidelinks]{hyperref}

% graphics %
\usepackage{graphicx}
\graphicspath{ {./images/} }

% environments %
\newtheorem{theorem}{Theorem}[section]
\newtheorem{corollary}{Corollary}[theorem]
\newtheorem{lemma}[theorem]{Lemma}

\theoremstyle{definition}
\newtheorem{definition}{Definition}[section]

\theoremstyle{remark}
\newtheorem*{remark}{Remark}
\newtheorem*{example}{Example}



% bibliography %
\bibliography{bibliography} 

\begin{document}

% title %
\title{From the Hot Universe to the Dark Matter}
\author{Buisine Léo\\Ecole Normale Superieure of Paris}
\maketitle

\tableofcontents

\chapter{Introduction}

Tout est sur le moodle \newline 
Examen: presenter un article comme si on l'avait fait, slides etc \newline 
Chaque cours est independant des autres. \par \medskip 

Dark universe = tout ce qu'on voit pas (neutrino, matiere noire, primordial black holes (PBH), etc...)

\chapter{Historical aspects of the dark matter}

Don't hesitate to read historical articles. \newline
It is important to smell the numbers \par \medskip 

\section{Some numbers}

The age of the (visible) universe: 13.8 billion years old (it is big and not big at the same time)  \newline
Cosmological principle: everyone in the universe sees the same universe. \newline 
The hubble law is the only law which preserves the cosmological principle\newline 
When was produced the CMB (cosmological microwave background)? 380 000 years (after the big bang) \newline 
What is the size of the Universe (radius of the universe): 93 billion light-years \newline 
Can galaxies recede faster than light? Yes. The galaxy doesn't feel like receding faster than light, but it can through the added effects of a warping/everchanging spacetime. There is no limit on the warping of spacetime \newline 
Cosmological horizon: horizon beyond which we cannot see light, because the metric extends so fast that the light doesn't move forward to us (the spacetime extends and pushes the light farther) \newline
Value of the Hubble constant ~ 70km/s/Mpc \newline
Anything at more than 10 Mpc recedes faster than they come to us to the gravity: they are not gravity bound. Andromeda is at 700kpc. In 40 million years, Everything which is not Andromeda nor us will cross the cosmological horizon. In 4 million years, the sun will explode but at approximately the same time, the Milky way will merge with Andromeda: it will become the Milkomeda \newline
A parsec is ~3 lightyears \newline 
Andromeda is the nearest galaxy, at around ~780kpc \newline
THe Milky way has a size of around 60 000 lyrs, for a mass of $10^12$ the mass of the sun \newline
Distance of the sun to the galaxy center: 8.5 kpc \newline 
Velocity of the earth around the sun: 30km/s \newline 
Velocity of the sun in the galaxy: 220km/s \newline 
Velocity of the Galaxy in the Local cluster: 650 km/s \newline 
Typical kinetic energy of a proton: $mv^2 ~ 1keV$ typical energy of a gas in a galaxy\newline 
Mean density of photon in the Universe: 411 cm$^{-3}$ (so $~10^90$ photons in the universe)\newline 
Density of protons around the sun 1 cm$^{-3}$ \newline 
Density of protons in the Milky way $10^{-3}$ cm$^{-3}$ \newline 
Density of protons between galaxies 1 m$^{-3}$ \newline 
Mean density of baryons in the Universe 2,4 $\times$ $10^{-17}$ cm$^{-3}$ \newline 
Density of DM around the sun $0.3$ GeV cm$^{-3}$ \newline 
Mean density of DM in the Universe $10^{-6}$ GeV cm$^{-3}$ \newline 
Mean density of neutrino in the Universe 112 cm$^{-3}$ \newline 
Mass of the Higgs boson 125 GeV \newline 
EW cross section $10^{-9}$ GeV$^{-2}$ \newline 
Actual limit on DM direct detection \newline 
$\sigma_{xp} \leq 10^{-46}$ cm$^{2}$ \newline 
Number of protons in 1 gram of matter $\mathcal N_A \simeq 10^24$ \newline 
Critical density $10^-5$ GeV cm$^{-3}$ \newline 
Scale + time of $\Lambda$ dominance $10Mpc$ \newline 
Density of proton in earth 1 g cm$^{-1}$ \newline 
1 GeV in Kelvin : $10^13$ K\newline 
Universe is 1000 times larger now than when the CMB was made. The size of the universe is inversely proportional to the temperature, the age of the universe is given by its temperature \newline 
Is energy conserved in the universe? No \par \medskip 

\section{History of observations}
Different bits in the history of the universe 
\begin{enumerate}
    \item The inflation era, dominated by the inflaton. Inflaton scale is a little bit more than the planck scale. The Hubble rate is constant, the universe is exponentially expanding. The universe is dominated by the dark energy.  
    \item The reheating era, where the inflaton is oscillating. Energy scale $10^19 \rightarrow 10^10$ GeV 
    \item The thermal era, dominated by radiation, with energy scales $10^10\rightarrow 1$ GeV 
    \item The neutrino and CMB era, from scale of the GeV to the eV, dominated by dark matter 
    \item Then the quintissence, our current era, dominated by dark energy once again, with constant hubble rate and with exponentially expanding universe. 
\end{enumerate}

We usually have $P$ the pressure, $\rho$ the density of energy, and the equation of state is usually written in term of $W = P/\rho$. For matter, $P=0$, and we get a dilution of the field $\rho \propto \frac{1}{a^3}$. For radiation, $W = 1/3$ and we get an even bigger dilution $\rho \propto \frac{1}{a^4}$ which also gives a redshift. For inflaton/quintissence, $W=-1$, and $\rho \propto 1$. The Hubble rate $H=\frac{\dot{a}}{a}$, we have $H^2 = \frac{\rho}{3 M_p^2}$
Inflaton scale is a little bit more than the planck scale. This is why the universe doesn't evolve the same way depending on the field considered. 

\subsection{Poincare point of view}

Proxima is the nearest sun, he knew the distance between the sun and proxima $R_s = 10^6 r_E$ where $r_E$ is the distance of earth to the sun. $\rho_s$ is the density of matter around the sun, $\rho_p$ the same around proxima. 
\begin{equation}
    \rho_s = \frac{M_s}{\frac{4}{3}\pi r_E^3}
\end{equation}
Writing the conservation of energy for the gravitational potential,
\begin{equation}
    v_E = \sqrt{\frac{8\pi G}{3}}\sqrt{\rho_S}r_E
\end{equation}
And we have the same for proxima around the galactic center. But noticing that in orders of magnitude, $v_s \sim v_E$, (where $v_S$ is the velocity of the sun and is approximately the same as the velocity of Proxima) we have 
\begin{equation}
    r_P = 10^9 r_E
\end{equation}
He thus computed the size of the galaxy. But knowing the density of stars (assuming the distance between the sun and proxima is an average distance between stars), he thus was able to compute the number of stars in the galaxy, $\sim 10^9$

\chapter{An expanding Universe}

The metric changes with time. Universe is almost homogeneous => only dependance on time. Dependance on space at a scale $10^{-5}$.
\section{The Friedmann equation}
This is the way of Milne. We suppose a gas made of non-relativistic particles of mass density $\rho_m$ and total energy per mass $k$. We can write the conservation of energy 
\begin{equation}
    E = m\phi(r) + \frac{1}{2}mv^2 = mk
\end{equation}
We could write a potential as in gravity $\phi = -\frac{GM}{r}$ or a potential as $\phi_{\Lambda} = \frac{\Lambda r^2}{6}$. Using the first potential, 
\begin{equation}
-\frac{GM}{r} + \frac{v^2}{2} = k   
\end{equation}
We can also write $v=\dot r$
\begin{equation}
    -\frac{GM}{r} + \frac{\dot{r}^2}{2} = k   
\end{equation}
We also write the mass density $M = \frac{4\pi}{3}r^3 \rho_m$, and the planck mass $M_p$ given by $G = \frac{1}{M_p^2}$. The reduced planck mass is $M_{pr} = \frac{M_p}{\sqrt{8\pi}}$. 

\begin{equation}
    \left(\frac{\dot r}{r}\right)^2 = H^2 = \frac{\rho_m}{3M_{pl}^2} - \frac{k}{r^2}
\label{eq:firsteq}\end{equation}
If we add the second potential, it also adds a "cosmological constant term", it adds a term in $\frac{\Lambda}{6}$. The first potential is attractive, and creates the Schwarchild radius $r_S = \frac{M}{M^2_{pl}}$. Any light inside the Schwarchild radius is kept inside the radius and cannot escape, due to the attractive force a gavity. On the other side, the second potential is repulsive, and corresponds to the expansion of the universe. It also lead to a radius, $r_? = ?$. Any light farther than this radius will never enter it. The light is so far that it cannot keep up with the expansion of the universe, and can never come close to us. \par \medskip 

We place ourselves in the comoving coordinates. $\chi$ is the factor of expansion, it is the length of a square which is appropriatly expanded with time. We take our position $r = R\times \chi = R_0 \times a(t) \times \chi$ with $R_0$ the radius of the universe at the beginning of time. We have 3 sources of matter in the universe, $\rho_m$ the matter, $\rho_\Lambda$ the cosmological constant (dark matter), and $\rho_R$ radiations, hence $\rho = \rho_m + \rho_\Lambda + \rho_R$. \par \medskip 

Once again, we write the conservation of the internal energy (with $V$ the volume, $P$ the pressure)
\begin{equation}
   \begin{aligned}
    U = \rho V,\quad \text{d}U &= \text{d}\rho ~V + \rho \text{d}V = -P\text{d}V \\
    &= \frac{4\pi}{3}\text{d}\rho~ R^3 + 4\pi R^2 \text{d}R ~\rho \dots 
   \end{aligned}
\end{equation}
And we end up finding 
\begin{equation}
    \dot \rho + 3H(\rho + P) = 0
\label{eq:eq2}\end{equation}
In this equation, we have no curvature but a pressure. It is nice since we don't have to know the curvature, but we need the pressure, in comparison with \eqref{eq:firsteq}. Derivating the hubble rate of \eqref{eq:firsteq} and inserting the above equation, 

\begin{equation}
    \dot H = -\frac{\rho + P}{2M_p^2} + \frac{k}{R^2}
\end{equation}
But since $H = \left(\frac{\dot R}{R}\right)^2 = \left(\frac{\dot a}{a}\right)^2$, we get 
\begin{equation}
    \dot H = \frac{\dot{\dot a}}{a} - \frac{\dot a^2}{a^2}
\end{equation}
But then 
\begin{equation}
    \frac{\dot {\dot a}}{a} = H^2 + \dot H = -\frac{\rho + 3P}{6M_p^2}
\end{equation}
Note that curvature is an output of the Hubble constant. \par \medskip

Einstein introduced the cosmological constant in order to cancel $\dot{\dot a}$, as to have a static universe. He did so because he observed a static universe. But Edington noticed that this would make an unstable universe. \par \medskip 

Equation of state: $P =w\rho$. For non-relativistic matter, $w = 0$. for the cosmological constant, $w = -1$. For relativistic particles, since the pressure is the quantity of momentum transferred on a unit of surface per second. For $E=pc$, $w = \frac{1}{3}$. For $E = \frac{1}{2}mv^2$, $w = \frac{2}{3}$. \par \medskip 

A better, Einstein based approach is given in the slides. We write some general curved metric, the Friedmann metric which only depends on the curvature. Then we try to solve the Einstein equations. \par \medskip 

From \eqref{eq:eq2}, we find 
\begin{equation}
    \rho \propto a^{-3(1 + w)}
\end{equation}
For dust, $\rho \propto a^{-3}$ is obvious since when the universe expand, the matter gets diluted. For radiation, $\rho \propto a^{-4}$ since the radiations gets diluted and redshifted, so they get diluted while loosing energy at the same time. For the cosmological constant, it stays constant. Notice that $k$ (the raw curvature, also possible to interpret as a field), evolves as $\rho \propto a^{-2}$. This means that at the beginning of time, there was no curvature. It is the flatness problem. The Hubble constant evolves as the square root of $\rho$, so 
\begin{equation}
    \dot a \propto a^{-\frac{1}{2} - \frac{3w}{2}} \Rightarrow a \propto t^{\frac{2}{3(1 + w)}}
\end{equation}
\chapter{The inflaton field}

L'inflaton est un champs qui depend uniquement du temps. On peut soit considerer un champs classique, soit prendre le champs classique pour un background field et faire un champs quantique par dessus, soit quantifier aussi le background field (typiquement la gravite quantique). \par \medskip 

The idea of Starobinski was to add a potential to the formula, creating the inflation. Slow-roll means $\phi$ is going slow, meaning the density is constant, meaning the universe expands a lot. \par \medskip 

The inflation tries to solve mainly 3 problems:
\begin{enumerate}
    \item Entropy problem: We have $10^90$ particles, how can we have so much entropy if there was no entropy at the big bang? (411 photons/$cm^3$, universe is $10^{29}$ cm long)
    \item Horizon (causality) problem : how can independant photons have the exactly same energy?
    \item Flatness problem
\end{enumerate}
The action for the inflaton
\begin{equation}
    \mathcal S = \int \text{d}^4 x \sqrt{-g}\left[\frac{1}{2}g^{\mu\nu} \partial_\mu \phi \partial_\nu \phi - V(\phi)\right]
\end{equation}
We have $\phi = \phi(t)$, the inflaton field acts as a Bose Einstein condensate. Everything acts as one, the field is constant, hence it is a classical field. There is no momentum, everything commutes. We want $\delta \mathcal S = 0$ so 

\chapter{Reheating}

\chapter{A thermal universe: FIMP and WIMP}

\chapter{BBN, CMB and warm dark matter}

\chapter{Direct detection}

\chapter{Indirect detection}

\chapter{Candidates}

\end{document}