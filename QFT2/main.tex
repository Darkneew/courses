\documentclass[a4paper]{book}

% packages % 
\usepackage[utf8]{inputenc} 
\usepackage{fvextra}
\usepackage{csquotes}
\usepackage[french, italian, spanish, english]{babel}
\usepackage[T1]{fontenc}   
\usepackage{color}  
\usepackage{amsmath, dsfont, amssymb, amsthm, stmaryrd}
\usepackage{slashed}
\usepackage[style=alphabetic]{biblatex}
\usepackage{enumitem}
\usepackage[hidelinks]{hyperref}

% graphics %
\usepackage{graphicx}
\graphicspath{ {./images/} }

% environments %
\newtheorem{theorem}{Theorem}[section]
\newtheorem{corollary}{Corollary}[theorem]
\newtheorem{lemma}[theorem]{Lemma}

\theoremstyle{definition}
\newtheorem{definition}{Definition}[section]

\theoremstyle{remark}
\newtheorem*{remark}{Remark}
\newtheorem*{example}{Example}



% bibliography %
\bibliography{bibliography} 

\begin{document}

% title %
\title{Quantum Field Theory 2}
\author{Buisine Léo\\Ecole Normale Superieure of Paris}
\maketitle

\tableofcontents

\chapter{Non-abelian gauge symmetry}

\section{Introducing the gauge field}

Let's consider a quantum field $\phi$ with the standard symmetry, transforming in a finite dimensional unitary representation $\mathcal U$ of a compact Lie group $G$. 
\begin{equation}
    \phi \rightarrow \mathcal U (g)\phi \qquad \text{for } g\in G
\end{equation}
Since it is finite dimensional, we can give the representation a finite index $i \in \mathbb{N}_d$
\begin{equation}
    \phi_i \rightarrow \mathcal U (g)_{ij}\phi_j \qquad \text{for } g\in G
\end{equation}
A potential term as $f(\phi^\dagger \phi)$ is invariant, as is a kinetic term $\partial_\mu \phi^\dagger \partial^\mu \phi$. \par \medskip 

What if $g$ varies over space-time (what if the symmetry acts locally instead of globally)?  
\begin{equation}
    \phi(x) \rightarrow \mathcal U (g(x))\phi(x) \qquad \text{for } g\in G
\end{equation}
The potential term is still invariant, but something happens to the kinetic term due to the derivative. 
\begin{equation}
    \partial_\mu \phi(x) \rightarrow \mathcal{U}(g(x)) \partial_\mu \phi(x) + \left[\partial_\mu \mathcal{U}(g(x))\right] \phi(x)
\end{equation}
We would like to promote the symmetry to a global one, but the default derivative doesn't seem to make it work in general. We would like to modify the derivative $\partial_\mu$ to $\mathcal D_\mu$ such that 
\begin{equation}
    \mathcal D_\mu \phi (x) \rightarrow \mathcal{U}(g(x)) \mathcal D _\mu \phi(x)
\end{equation}
That is we want to take into account the symmetry in the geometry, or modify the symplectic structure to take into account the gauge degrees of freedom, or take secretly into account the coupling to the degrees of freedom in the kinetic term. To do so, we introduce a new field (the gauge field) $A_\mu$ and write\footnote{the conventions will change}
\begin{equation}
    \mathcal D _\mu = \partial_\mu - A_\mu
\end{equation}
We have 
\begin{equation}
    \mathcal D' _\mu \phi'(x) = (\partial_\mu - A'_\mu) \mathcal U (g)\phi = \mathcal U (g) \partial_\mu \phi + \left[\partial_\mu \mathcal{U}(g)\right] \phi - A'_\mu\mathcal U (g)\phi 
\end{equation}
But we want it to be equal to 
\begin{equation}
    \mathcal U(g) \mathcal D_\mu \phi = \mathcal U(g)(\partial_\mu - A_\mu )\phi
\end{equation}
Such that 
\begin{equation}
    \mathcal U (g)A_\mu = A'_\mu \mathcal U(g) - \partial_\mu \mathcal U (g)
\end{equation}
In other terms
\begin{equation}
    A'_\mu = \mathcal U (g)A_\mu\mathcal U ^{-1}(g) + \left[\partial_\mu \mathcal U (g)\right] \mathcal U ^{-1}(g)
\end{equation}
What kind of object is $A_\mu$? Using a matrix realization of the Lie algebra, considering 
\begin{equation}
    \text{exp} : \mathfrak g \rightarrow G
\end{equation}
We write (at least close to the identity) 
\begin{equation}
    \mathcal U(g) = e^{i\alpha^i(g)\tau_i}
\end{equation}
with $\tau^i$ a basis of $\mathfrak{g}$ in appropriate representation. 
\begin{equation}
    \left[\partial_\mu \mathcal U (g)\right] \mathcal U ^{-1}(g) = \left[\partial_\mu \alpha^i (g)\right]\tau_i
\end{equation}
So $A_\mu$ is Lie algebra valued, $A_\mu = A_\mu^i \tau_i$. Following physics conventions, we write the generators of the Lie algebra as $i \tau_i$ (since $\mathcal U$ is unitary, the $\tau_i$ are then hermitians, which is likeable). So the convention changes for $\mathcal D_\mu$
\begin{equation}
    \mathcal D_\mu = \partial_\mu - iA_\mu \quad \text{with  } A_\mu = A_\mu^i \tau_i
\end{equation}
Substituting using the new convention, we have 
\begin{equation}
    A'_\mu = \mathcal U (g)A_\mu\mathcal U ^{-1}(g) - i\left[\partial_\mu \mathcal U (g)\right] \mathcal U ^{-1}(g)
\end{equation}
Note that 
\begin{equation}
    \tau \rightarrow \mathcal U \tau \mathcal U ^{-1}
\end{equation}
is the adjoint action of the Lie group on the Lie algebra. \par \medskip 

Now, let's rewrite things infinitesimally, for $\alpha = \alpha^i\tau_i$. 
\begin{equation}
    \mathcal U (g) A_\mu \mathcal U^{-1}(g) = A_\mu + i[\alpha, A_\mu] + O(\alpha^2)
\end{equation}
and 
\begin{equation}
    - i\left[\partial_\mu \mathcal U (g)\right] \mathcal U ^{-1}(g) = \partial_\mu \alpha^i(g)\tau_i
\end{equation}
Such that 
\begin{equation}
    A'_\mu = A_\mu + \partial_\mu \alpha + i[\alpha, A_\mu ]
\end{equation}
In compontents\footnote{There is no meaning to the height of the indices. They can be raised or lowered at will},
\begin{equation}
    \begin{aligned}
        A^k_\mu \tau_k &\rightarrow A^k_\mu \tau_k + \partial_\mu \alpha^k \tau_k + i\alpha^i A^j_\mu [\tau_i, \tau_j] \\
        &\rightarrow A^k_\mu + \partial_\mu \alpha^k - \alpha^i A^j_\mu C^k_{ij}
    \end{aligned}
\end{equation} 
Notice how the last term vanishes when the Lie algebra is abelian. \par \medskip 

$A^K_\mu$ is called the gauge field, $\mathcal D_\mu$ the gauge covariant derivative. An action (without $A_\mu$) with global symmetry group $G$ (ie the fields transform in a unitary representation of $G$) becomes invariant under a local symmetry with gauge group $G$ upon replacing $\partial_\mu \rightarrow \mathcal D_\mu$. This introduces the gauge field into the action. 

\section{The kinetic term of the gauge field}

Looking for gauge invariant $2^{nd}$ order in derivatives quadratic term in $A_\mu$. Consider $\mathcal D_\mu \mathcal D_\nu \phi$ for any field $\phi$ transforming in some representation of $G$. 
\begin{itemize}
    \item Includes a derivative of $A^\mu$
    \item transforms nicely 
\end{itemize}
How do we get rid of $\phi$ in this term?
\begin{equation}
    \mathcal D_\mu \mathcal D_\nu \phi = \partial_\mu \partial_\nu \phi  - i(\partial_\mu A^k_\nu)\tau_k \phi - i  A^k_\nu \tau_k \partial_\mu \phi - i A^i_\mu\tau_i \partial_\nu \phi - A^i_\mu A^k_\nu \tau_i \tau_k \phi
\end{equation}
We can try to consider the commutator of the covariant derivatives 
\begin{equation}
    [\mathcal D_\mu, \mathcal D_\nu]\phi = \left(-i (\partial_\mu A^j_\nu - \partial_\nu A^j_\mu) - iA^i_\mu A^k_\nu C^j_{ik}\right)\tau_j \phi
\end{equation}
Hence $[\mathcal D_\mu, \mathcal D_\nu]$ is a matrix operator, in contrast to a derivative operator. We define 
\begin{equation}
    F_{\mu\nu} = i[\mathcal D_\mu, \mathcal D_\nu] 
\end{equation}
the field strength, for $\mathcal D_\mu$ in some representation. 
\begin{equation}
    F_{\mu\nu} = F_{\mu\nu}^k \tau_k = \left(-i (\partial_\mu A^k_\nu - \partial_\nu A^k_\mu) - iA^i_\mu A^j_\nu C^k_{ij}\right)\tau_k
\end{equation}
How does $F_{\mu\nu}$ transforms? 
\begin{equation}
    \mathcal D'_\mu \mathcal D'_\nu \phi' = \mathcal U (g) \mathcal D_\mu \mathcal D_\nu \phi 
\end{equation}
So 
\begin{equation}
    F'_{\mu\nu} = \mathcal U (g) F_{\mu\nu} \mathcal U^{-1}(g)
\end{equation}
Hence $\text{Tr}(F_{\mu\nu}F_{\rho\sigma})$ is gauge invariant.
\begin{remark}
    To define this product, we can either work in the universal envelopping algebra or in any representation.
\end{remark}

Now, there are two Lorentz-invariant contractions. 
\begin{enumerate}
    \item $\text{Tr}(F_{\mu\nu}F^{\mu\nu})$ : it is the kinetic term
    \item $\text{Tr}(F_{\mu\nu}F_{\rho\sigma})\varepsilon^{\mu\nu\rho\sigma}$ : it will play a role later. Notice that it is a total derivative
\end{enumerate}

\section{Assorted facts about Lie algebras}
\subsection{The trace bilinear}

\begin{equation}
    \text{Tr}(F_{\mu\nu}F^{\mu\nu}) = F_{\mu\nu}^{~~~k}F^{\mu\nu,l}\text{Tr}(\tau_k\tau_l)
\end{equation}
To have a well defined kinetic term, we need $\text{Tr}(\tau_k\tau_l)$ to be non-degenerate and positive or negative definite. In the case of the Standard model, we can fix a specific representation of $su(2), su(3), \dots$ and check the relations. Towards negative definites statement of $\text{Tr}(\tau_i\tau_j)$ 
\begin{definition}
Given a represent. $(\pi, V)$ of $\mathfrak{g}$, the bilinear form $B_V(.,.) = \text{Tr}(\pi(.)\pi(.))$ is the trace bilinear form with respect to $V$. 
\end{definition}
With respect to the adjoint representation, it is the Killing form. 
\begin{theorem}
    The killing form is non-degenerate iff $\mathfrak g$ is non-degenerate. 
\end{theorem}
\begin{theorem}
    Given a semisimple $\mathfrak g$, its Killing form is negative definite iff its Lie group is compact
\end{theorem}

\subsection{Normalisation of the trace bilinear}
\begin{definition}
    Let $(\pi, V)$ be a rep of $\mathfrak g$ . A bilinear form $B$ on $V$ is invariant wrt $\pi$ if for $x\in \mathfrak g$, for $v,w\in V$, 
    \begin{equation}
        B(\pi(x)v, w) + B(v, \pi(x)w) = 0
    \end{equation}
\end{definition}
Proposition: the trace bilinear is invariant with respect to the adjoint rep. \newline 
Proposition: Let $\mathfrak g$ be a finite dimensional simple Lie algebra. Then there exist, up to a scalar, at most one invariant bilinear form. \par \medskip 

Since the adjoint action always give a invariant bilinear form in any representation, we end up with the fact that our trace bilinear must be the Killing form up to a scalar. The \underline{Dynkin index} $T(\pi)$ is the scalar in front of the bilinear (or a rescaled version of it). 

\subsection{Relating the Dynkin index to the Casimir of the representation}
Let $\mathfrak g$ be a semisimple Lie algebra, $\{x_i\}$ be a basis of $\mathfrak{g}$, $\{y_i\}$ be a dual basis of $\mathfrak{g}$ with respect to the Killing form. The Dynkin index $T(\pi)$ is defined as the scalar relating the trace bilinear in the representation $\pi$ to the Killing form. \par \medskip 

Prop: $J = \sum x_i y_i$ commutes with all elements of $\mathfrak{g}$. It is independant of the choice of basis. $J$ is called the quadratic Casimir element of $\mathfrak{g}$. Let $(\pi, V)$ be an irreducible representation of $\mathfrak{g}$. $\pi(J)$ commutes with all elements of $\pi(\mathfrak{g})$. By Schur's lemma, $\pi(J) = C(\pi)\text{Id}$, with $C(\pi)\in \mathbb{C}$. $C(\pi)$ is called the Casimir of the representation. By a quick computation, we see that 
\begin{equation}
    C(\pi)\times \frac{d}{n} = T(\pi)
\end{equation}
with $d$ the dimension of the representation and $n$ the dimension of the Lie algebra. 

\subsection{Exemple $SU(N)$}

Lie algebras defined as matrix subalgebras of $\mathfrak{gl}(n)$ come with a canonical representation on $\mathbb{C}^n$, which is the defining representation. \newline 
NB: Physics convention: replace Killing form in definition of the Dynkin index by 
\begin{equation}
    2\text{Tr}_{\pi_0}
\end{equation}
where $(\pi_0, V_0)$ is the defining representation. Hence the Dynkin index of the defining representation is $\frac{1}{2}$ in these conventions. In these conventions we also define the Casimir element using dual basis with respects to the inner product $2\text{Tr}_{\pi_0}$. \par \medskip 

For explicit computation, we can introduce an orthonormal basis $\{T_i\}$ of the defining representation of $\mathfrak{su(N)}$ with regards to $2\text{Tr}$. 
\begin{equation}
    \begin{aligned}
        &\text{Tr}(T_iT_j) = \frac{1}{2}\delta_{ij} \\
        &\text{Tr}((\pi(T_i)\pi(T_j)) = T(\pi)\delta_ij
    \end{aligned}
\end{equation}
With respect to $2\text{Tr}_{\pi_0}$, $\{T_i\}$ is a self-dual basis. 
\begin{equation}
    \dim \pi_0 = N 
\end{equation}
On the other hand, 
\begin{equation}
    \dim (\mathfrak{su}(N)) = \#\{\text{basis of traceless hermitian matrices}\} = N^2 - 1
\end{equation}
Such that 
\begin{equation}
    C(\pi_0) = \frac{N^2-1}{2N}
\end{equation}
Note: the structure constants in such a basis are completely antisymmetric. 
\begin{equation}
    [T_i,T_j] = iC_{ij}^kT_k
\end{equation}
with $f_{ijk} = C_{ij}^k$, $f$ is completely antisymmetric

\section{Quantizing gauge theories}

By construction of the gauge symmetry, the second order differential operator defininf the kinetic term of the gauge field is not invertible. We need to gauge fix to define the propagator. But what is left of the gauge fixing, that can help us fix the renormalization counterterms? 

\subsection{The Fadeev-Popov procedure}

Recall 
\begin{equation}
    F_{\mu\nu} = \partial_\mu A_\nu - \partial_\nu A_\mu - i[A_\mu, A_\nu]
\end{equation}
Then $\text{Tr}(F_{\mu\nu}F^{\mu\nu})$ gives kinetic terms plus interaction terms. In an orthogonal basis, $\text{Tr}(T_iT_j)\propto \delta_{ij}$ so the gauge kinetic term is $\dim \mathfrak{g}$ copies of the photon kinetic term 
\begin{equation}
    A_i^\mu (g_{\mu\nu}\square - \partial_\mu\partial_\nu)A^\nu_i
\end{equation}
Indeed, for $A^\mu_i$ an eigenvector of the momentum operator, we get $(g_{\mu\nu}k^2 - k_\mu k_\nu)$ which has a non trivial kernel. However, we could eliminate this kernel by shifting the coefficient of $\partial_\mu\partial_\nu$ away from $-1$: 
\begin{equation}
    \begin{aligned}
        &\mathcal{L}[A, \phi_i] + \frac{1}{2\xi} A_i^\mu \partial_\mu\partial_\nu A^\nu_i \\
        &= \mathcal{L}[A, \phi_i] - \frac{1}{2\xi} (\partial_\mu A^\mu_i)(\partial_\nu A^\nu_i) + \text{ total derivatives}
    \end{aligned}
\end{equation}
How can we introduce a term $e^{-\frac{i}{2\xi}\int \text{d}^4x(\partial_\mu A^\mu_i)(\partial_\nu A^\nu_i)}$ into the action? The strategy is to multiply the partition function $Z$ action by field independant terms which do not change the amplitude. \par \medskip 

\begin{equation}
    e^{-\frac{i}{2\xi}\int \text{d}^4x(\partial_\mu A^\mu_i)(\partial_\nu A^\nu_i)} = \int \mathcal Df e^{-\frac{i}{2\xi}\int \text{d}^4x ff} \delta(\partial_\nu A^\nu_i - f)
\end{equation}
We write $B_\xi[f] = e^{-\frac{i}{2\xi}\int \text{d}^4x ff}$, and $G[A] = \delta(\partial_\nu A^\nu_i - f)$. We also write $A^g$ the field $A$ transformed by the action of $g$. Integrating over the gauge group, we trivially have 
\begin{equation}
    1 = \int \mathcal D g \delta(G[A^g]) \det \frac{\delta G[A^g]}{\delta g}
\end{equation}
Now 
\begin{equation}
    \int \mathcal D g \int \mathcal D f B_\xi[f] \delta(G[A^g]) \det \frac{\delta G[A^g]}{\delta g} = C(\xi) 
\end{equation}
does not depend on $A$. Let's rescale 
\begin{equation}
    Z = \int \mathcal D A \mathcal D \phi e^{iS[A, \phi]}
\end{equation}
into $Z' = C(\xi) Z$
\begin{equation}
    Z' = \int \mathcal D g \int \mathcal D f\int \mathcal D A \mathcal D \phi ~e^{iS[A, \phi]} B_\xi[f] \delta(G[A^g]) \det \frac{\delta G[A^g]}{\delta g} 
\end{equation}
We want to use gauge invariance to replace dependance on $A^g$ by dependance on $A$. First note 
\begin{equation}
    \frac{\delta G[A^{g\tilde g}]}{\delta \tilde g} \Bigg|_{\tilde g = e} = \frac{\delta G[A^{g}]}{\delta g} \frac{\delta \tilde g g}{\delta \tilde g}\Bigg|_{\tilde g = e}
\end{equation}
Such that 
\begin{equation}
    \det \frac{\delta G[A^g]}{\delta g} = \det \frac{\delta G[A^{g\tilde g}]}{\delta \tilde g} \Bigg|_{\tilde g = e} \left[\det \frac{\delta \tilde g g}{\delta \tilde g}\Big|_{\tilde g = e}\right]^{-1}
\end{equation}
We can write $\mu(g) = \left[\det \frac{\delta \tilde g g}{\delta \tilde g}\Big|_{\tilde g = e}\right]^{-1}$. 
\begin{equation}
    Z' = \int \mathcal D g \mu(g) \mathcal D f \mathcal D A \mathcal D \phi ~e^{iS[A, \phi]} B_\xi[f] \delta(G[A^g]) \det \frac{\delta G[A^{g\tilde g}]}{\delta \tilde g} \Bigg|_{\tilde g = e} 
\end{equation}
But if the gauge symmetry is preserved by the quantum theory, the measure should be invariant under a gauge transform. So we can integrate over $A^{g^{-1}}$ instead of over $A$, and (necessarily) similarly for $\phi$. So 
\begin{equation}
    Z' = \int \mathcal D g \mu(g) \mathcal D f \mathcal D A^{g^{-1}} \mathcal D \phi^{g^{-1}} e^{iS[A^{g^{-1}}, \phi^{g^{-1}}]} B_\xi[f] \delta(G[A]) \det \frac{\delta G[A^{\tilde g}]}{\delta \tilde g} \Bigg|_{\tilde g = e} 
\end{equation}
The $g$ dependance has factored out into a coefficient 
\begin{equation}
    \int \mathcal D g \mu(g) = \int \mathcal D g \det^{-1} \frac{\delta \tilde g g}{\delta \tilde g}\Big|_{\tilde g = e} = V_G
\end{equation}
which can be easily interpreted as the volume of the Gauge group. Indeed, integrating over all of the possible fields before gauge fixing should yield a factor $V_G$. Since the gauge group is unphysical, we should have divided at some point the whole partition function by $V_G$ to take into account the gauge redundancy. We are happy to see that $V_G$ naturally decouples from $Z'$. We also notice that $V_G$ naturally generalizes the Haar measure of the Lie group to a gauge group. We can define 
\begin{equation}
    Z = \frac{\int \mathcal D A \mathcal D \phi~ e^{iS}}{V_G}
\end{equation}
We would now like to reformulate the other not nice term, to get it in a nice shape and understand its reason for appearing. 

\begin{equation}
    \det \frac{\delta G[A^{ g}]}{\delta  g} \Bigg|_{ g = e} = \det \frac{\delta [\partial_\mu A^{g,\mu} - f]}{\delta  g} \Bigg|_{ g = e}
\end{equation}
To evaluate this derivative around $g=e$, we can consider infinitesimal gauge transformations 
\begin{equation}
    A_\mu^{e+\pi} = A_\mu + \mathcal D _\mu \pi = A_\mu + \partial_\mu \pi - i[A_\mu, \pi]
\end{equation}
Such that 
\begin{equation}
    \frac{\delta G[A^{e+\pi}]}{\delta \pi} \Bigg|_{ \pi = 0} = \partial_\mu \mathcal D _\mu
\end{equation}

Identity from fermionic path integral: 
\begin{equation}
    \det \mathcal O = \int \text{d}\bar \psi \text{d} \bar \psi ~e^{-i \int \text{d}^4x \bar\psi \mathcal O \psi} 
\end{equation}
Such that 
\begin{equation}
    \det \partial_\mu \mathcal D _\mu = \int \mathcal D\bar{c}\mathcal D c~ e^{i\int \text{d}^4 x \bar c (-\partial_\mu \mathcal D _\mu)c}
\end{equation}
Such that we end up with new fermionic fields 
\begin{equation}
    Z' = \int \mathcal D A \mathcal D \phi \mathcal D \bar c \mathcal D c e^{i\int \text{d}^4 x \left[\mathcal L [A, \phi] - \frac{1}{2\xi}(\partial_\mu A^\mu _i )^2 - \bar c_i \partial_\mu D^\mu c_i\right]}
\end{equation}
Note the fields $c_i, \bar c_i$: 
\begin{itemize}
    \item Are fermionic, to obtain the correct determinant
    \item Lorentz scalars in the adjoint representation, as must have same quantum number as $\pi$
    \item They break the spin statistics theorem, and hence cannot appear in physical states. They cannot enter in/out states. 
    \item This justify their names, ghosts and anti ghosts
\end{itemize}

We will study the Feynman rules that govern such theories presently. Consider again 
\begin{equation}
    Z'/V_G = = \int \mathcal D f \mathcal D A \mathcal D \phi e^{iS[A, \phi]} B[f] \delta(G[A]) \det \frac{\delta G[A^{ g}]}{\delta  g} \Bigg|_{ g = e} 
\end{equation}
Note: $Z$ and $Z'$ are related by a field independant constant for any choice of $B[f]$, $G[A]$. It reflects the freedom to choose any gauge we want. The choice 
\begin{equation}
    B_\xi[f] = e^{-\frac{i}{2\xi}\int \text{d}^4x f_if^i}\qquad G[A] = \delta (f - \partial_\nu A^\nu_i )
\end{equation}
are called a generalized $\xi$-gauges. How is it related to gauge fixing? A generalization thereof, but it reduces to standard notion in the limit $\xi \rightarrow 0$ limit. In this case, we can evaluate the $\mathcal D f$ integral via stationary phase ($B[f]$ oscillates rapidly except at f =0), where $ \delta (f - \partial_\nu A^\nu_i )$ becomes $ \delta (\partial_\nu A^\nu_i )$, imposing the Lorenz gauge. 

\subsection{BRST symmetry}
\subsubsection{The Fadeev Popov Lagrangian is BRST invariant}

Symmetries constrain the form of counterterms required to renormalize the theory. What about gauge fixed gauge symmetries? 

\begin{equation}
    L_{FP} = L[A, \phi] - \frac{1}{2\xi}(\partial_\mu A^\mu_i)(\partial_\nu A^\nu_i) + \partial_\mu \bar{c}_i \mathcal D^\mu c_i
\end{equation}
The usual Lagrangian is invariant under gauge transformations, whilst $L_{GF} = - \frac{1}{2\xi}(\partial_\mu A^\mu_i)(\partial_\nu A^\nu_i) + \partial_\mu \bar{c}_i \mathcal D^\mu c_i$ corresponds to the gauge fixing Lagrangian. We would like to recover some derivative. What if we do a gauge transformation proportional to $c$? Actually, since $c$ is fermionic, we can't do it directly. But that is the general idea of BRST symmetry. Let's consider a gauge transformation of $L_{FP}$ with gauge parameter $\alpha(x) = \theta c(x)$, with $\theta$ a grassmanian variable and $c(x)$ a ghost which is Lie algebra valued, ie $c(x) = c^i(x)\tau_i$
\begin{equation}
    \begin{aligned}
        A^\mu_i &\rightarrow A^\mu_i + \theta D^\mu c_i \\
        - \frac{1}{2\xi}(\partial_\mu A^\mu_i)(\partial_\nu A^\nu_i) &\rightarrow - \frac{1}{2\xi}(\partial_\mu A^\mu_i)(\partial_\nu A^\nu_i) - 2\frac{\theta}{2\xi} (\partial_\mu A^\mu_i)(\partial_\nu D^\nu c_i)
    \end{aligned}
\end{equation}
How should $\bar{c}_i, c_i$ transform to make $L_{FP}$ invariant? 
\begin{equation}
    \bar{c}_i \rightarrow \bar{c}_i - \frac{1}{\xi} \theta \partial_\mu A^\mu_i
\end{equation}
is almost good enough, except that the gauge covariant derivative in front of $c_i$ also transforms under the transformation due to its dependance in $A_\mu$, messing things up by a little bit. 
\begin{equation}
    \begin{aligned}
        c &\rightarrow c + \theta \delta c \\ 
        A^\mu &\rightarrow A^\mu + \theta D^\mu c\\
        \mathcal D^\mu c = \partial^\mu c - i[A^\mu, c] &\rightarrow D^\mu c + \theta \partial^\mu \delta c - i[A^\mu, \theta \delta c ] - i [D^\mu \theta c, c]
    \end{aligned}
\end{equation}
Where 
\begin{equation}
    D^\mu \theta c = \theta \partial^\mu c + i[\theta c, A^\mu]
\end{equation}
We need 
\begin{equation}
    \theta (\partial^\mu \delta c - i[A^\mu, \delta c ]) -i[\theta \partial^\mu c, c] + [[\theta c, A^\mu], c] = 0
\end{equation}
We just have to solve this for $\delta c$! We find 
\begin{equation}
    \delta c = \frac{i}{2} c_i c_j [\tau^i, \tau^j]
\end{equation}
which guarantees that 
\begin{equation}
    \mathcal D_\mu c \rightarrow \mathcal D_\mu c
\end{equation}
Hence, we found a (fermionic) symmetry of the Lagrangian, the BRST symmetry.

\begin{equation}
    \begin{aligned}
        &\delta_{\text{BRST}}A^\mu = \mathcal D^\mu \theta c \\ 
        &\delta_{\text{BRST}} \phi = \delta_{\theta c}^{\text{gauge} }\phi = i \theta \pi(c) \phi \\ 
        &\delta_{\text{BRST}} \bar{c} = -\frac{1}{\xi} \theta \partial_\mu A^\mu \\
        &\delta_{\text{BRST}} c = \theta  \frac{i}{2} c_i c_j [\tau^i, \tau^j] = \theta  \frac{i}{2} \{c_i, c_j\} 
    \end{aligned}
\end{equation}
for some representation $\pi$. Hence, the only counterterms required to renormalize this Lagrangian are BRST invariant. In particular, terms not involving ghosts are gauge invariants. 

\subsection{BRST without FP}
We will argue 
\begin{enumerate}
    \item There exist an operator $Q_{\text{BRST}}$ such that for $\phi$, $\phi_i$ physical fields (ie not ghosts?gauge?x?) 
    \begin{equation}
        \langle [Q_{\text{BRST}}, \phi ] \pi \phi_i\rangle = 0
    \end{equation}
    \item The gauge fixing part of the Lagrangian is of the form $[Q_{\text{BRST}}, \phi ]$. 
\end{enumerate}
Consider the transformation $\delta_{\text{BRST}}$ introduced above, but introduce an additional field $N = N_i \tau^i$ and replace $\delta_{\text{BRST}} \bar{ c} = \theta N$, $\delta_{\text{BRST}} N = 0$. We also introduce a multiplicative\footnote{the ghost number of the product of terms is the sum of the ghost numbers} grading called the ghost number: $c$ has a ghost number 1, $\bar c$ has a ghost number -1, and all other fields 0. We write 
\begin{equation}
    \delta_{\text{BRST}} \psi = \theta \triangle \psi
\end{equation}
Where $\triangle$ is the Slavnov operator. This operator increases the ghost number by one (except when it annihilates). 

\begin{remark}
    $\triangle$ is a graded differential, ie 
    \begin{equation}
        \triangle(\phi \psi) = (\triangle \phi)\psi + (-1)^{\text{ghost number of }\phi}\phi\triangle\psi
    \end{equation}
\end{remark}

$\triangle ^2 = 0$, ie $\triangle$ is nilpotent (this is why we had to introduce $N$). Hence it acts as a differential with respect to the grading by the ghost number. It is also closed. Hence suppose we modify the action before gauge fixing by a BRST exact term 
\begin{equation}
    \mathcal L \rightarrow \mathcal L + \triangle \psi 
\end{equation}
By gauge invariance of $\mathcal L$ and nilpotency of $\triangle$, this modified Lagrangian density is BRST invariant. Let's introduce a charge $Q_{\text{BRST}}$ that implements the BRST transformation. 
\begin{equation}
    [Q_{\text{BRST}}, \psi]_{-s} =  \triangle \psi 
\end{equation}
where $[.,.]_{-1}$ is the anticommutator and $[.,.]_{+1}$ is the commutator, with $s = (-1)^{\text{ghost number of }\psi}$. \par\medskip 

Assume a BRST invariant vacuum exists. Then 
\begin{equation}
    \langle 0| [Q_{\text{BRST}} , T{\pi\phi_i}]_{-s = \pi s_i} |0\rangle 
\end{equation}
Also assuming that $\phi_j$ with $j = 2, \dots n$ are BRST closed, ie $[Q_{\text{BRST}}, \phi_j]_{-s_j} = 0$, then 
\begin{equation}
    \langle 0| T(\phi \prod_{k\neq 1}\phi_k)|0\rangle = 0
\end{equation}
correlators of BRST closed operator with a BRST exact operator vanish. Hence. the modification 
\begin{equation}
    \mathcal L \rightarrow \mathcal L + \delta \mathcal L 
\end{equation}
with $\delta \mathcal L = [Q_{\text{BRST}}, \psi]_+ = \triangle \psi$ for any operator $\psi$ of ghost number -1 leave correlators of BRST invariant operators invariant. In particular, BRST invariant means gauge invariant for operators not involving ghosts. \par \medskip 

We pick $\psi = \bar{c_i}(F_i + \frac{1}{2}\xi N_i)$, with $F_i$ a functional of ordinary fields (fields with ghost number 0 except $N$). Then 
\begin{equation}
    \delta \mathcal L = \triangle \psi = N_i F_i - \int \text{d}^4 y ~\bar{c}_i(x) \frac{\delta F_i}{\delta \phi_A(y)}[Q_{\text{BRST}}, \phi_A (y)]_- + \frac{1}{2}\xi N_i^2
\end{equation}
where the index $A$ runs over all ordinary fields. We can integrate out $N$ by completing the spinor/sphere/?? 
\begin{equation}
    \frac{1}{2}\xi N_i^2 + N_i F_i = \frac{1}{2}\xi (N_i + \frac{1}{\xi}F_i)^2 - \frac{1}{2}\frac{1}{\xi}F_i^2
\end{equation}
such that 
\begin{equation}
    \delta \mathcal L = - \int \text{d}^4 y ~ \bar{c}(x)\frac{\delta F_i(x)}{\delta \phi_A(y)}[Q_{\text{BRST}}, \phi_A(y)] - \frac{1}{2\xi}F_i^2
\end{equation}
Notice that $[Q_{\text{BRST}}, \phi_A(y)] \propto c$. The first term is the kinetic term for ghosts. \par \medskip 

For exemple, for $F_i = \partial_\mu A^\mu _i$:
\begin{equation}
    \delta \mathcal L = \partial_\mu \bar{c}(x) \mathcal D ^\mu c(x) - \frac{1}{2\xi}(\partial_\mu A^\mu_i)^2 = \mathcal L _{GF} - \frac{1}{2\xi}(\partial_\mu A^\mu_i)^2 
\end{equation}
We recall that gauge invariant correlators are not modified by such a change in the Lagrangian. Note that we arrived at this result without invoking the FP procedure. 

\section{Perturbation theory of Yang-Mills theory}

\subsection{Feynman rules}
\subsubsection{Introducing the gauge coupling}

To introduce gauge coupling, replace 
\begin{equation}
    \mathcal D_\mu = \partial_\mu - i A_\mu \rightarrow \mathcal D_\mu = \partial_\mu - i g A_\mu 
\end{equation}
which can be seen as just renormalizing $A_\mu$. This change changes the gauge transformation. Before, we had 
\begin{equation}
    A'_\mu = \mathcal U(h)A_\mu \mathcal U^{-1}(h) - i\partial_\mu \mathcal{U}(h) ~ \mathcal{U}(h)^{-1}
\end{equation}
But now,
\begin{equation}
   g A'_\mu = g\mathcal U(h)A_\mu \mathcal U^{-1}(h) - i\partial_\mu \mathcal{U}(h) ~ \mathcal{U}(h)^{-1}
\end{equation}
Which means 
\begin{equation}
    A'_\mu = \mathcal U(h)A_\mu \mathcal U^{-1}(h) - \frac{i}{g}\partial_\mu \mathcal{U}(h) ~ \mathcal{U}(h)^{-1}
\end{equation}
Infinitesimally, 
\begin{equation}
    A'_\mu = A_\mu + \frac{1}{g}\partial_\mu \alpha + i [\alpha, A_\mu ]
\end{equation}
To remove the $g$ dependance of derivative term in $F^{\mu\nu}$, redefine 
\begin{equation}
    F_{\mu\nu} = \frac{i}{g}[\mathcal D_\mu, \mathcal D_\nu] = \partial_\mu A_\nu - \partial_\nu A_\mu - ig[A_\mu, A_\nu]
\end{equation}

\subsubsection{The Lagrangian}
\begin{equation}
        \begin{aligned}
            \mathcal L =& -\frac{1}{4} \sum_{a} (F_a)_{\mu\nu}(F_a)^{\mu\nu} - \frac{1}{2\xi}\sum_a (\partial_\mu A_a^{~\mu})^2 + \partial_\mu \bar{c}_a (\delta_{ac}\partial^{\mu} + gf_{abc}A_b^{~\mu})c_c \\
            &+ \bar{\psi_i}(\delta_{ij} i\slashed{\partial} + g\slashed{A}_aT_{ij}^a - m\delta_{ij})\psi_j  \\
            &+ [(\delta_{ki}\partial_\mu - ig (A_a)_\mu T^a_{ki})\phi_i]^*[(\delta_{kj}\partial^\mu - ig A^\mu_a T^a_{kj})\phi_j] - M^2 \phi^*_i\phi_i
        \end{aligned}
\end{equation}
The indices in greek letters are Lorentz indices, the beginning of the alphabet indices are Lie indices, and the middle of the alphabet indices are field indices. The $T$ are the representation of the Lie algebra in our theory. We can divide this Lagrangian into several parts. The kinetic term is 
\begin{equation}
    \begin{aligned}
        \mathcal L_{\text{kin}} =& -\frac{1}{4}(\partial_\mu (A_a)_\nu - \partial_\nu (A_a)_\mu)(\partial^\mu (A_a)^\nu - \partial^\nu (A_a)^\mu) \\
        &-\frac{1}{2\xi}\sum_a (\partial_\mu A_a^{~\mu})^2 + \bar \psi_i (i \slashed{\partial} - m)\psi_i \\ 
        &- \phi^*_i (\square + M^2)\phi_i - \bar c_a \square c_a 
    \end{aligned}
\end{equation}

\subsubsection{Propagators}
The Gauge boson. In momentum space, 
\begin{equation}
    \frac{1}{4}(p_\mu A_\nu - p_\nu A_\mu)^2 + \frac{1}{2\xi}(p_\mu A^\mu)^2 = -\frac{1}{2}A^a_\mu (-p^2g^{\mu\nu} + (1 - \frac{1}{\xi})p^{\mu\nu})\delta_{ab}A^b_\nu
\end{equation}
To find the propagator, we make the ansatz 
\begin{equation}
    \Pi^A_{\mu\nu} = A(p^2)p_{\mu}p_{\nu} + B(p^2)g_{\mu\nu}
\end{equation}
Writing $\Pi^A_{\mu\nu}(-p^2g^{\mu\nu} + (1 - \frac{1}{\xi})p^{\mu\nu}) = g_\mu^\rho$ gives 
\begin{equation}
    A = \frac{-(1 - \frac{1}{\xi})\frac{1}{p^2}}{\frac{1}{\xi}p^2} \qquad B = -\frac{1}{p^2}
\end{equation}
Which results in 
\begin{equation}
    i(\Pi^A_{ab})_{\mu\nu} = -i \frac{g_{\mu\nu}- (1-\xi)\frac{p_\mu p_\nu}{p^2}}{p^2 + i\epsilon} \delta_{ab}
\end{equation}
We draw the propagator as a bubbly line? as a cloud line going in spirals, labelled by a Lie index and a Lorentz index. For ghosts, we have 
\begin{equation}
  i\Pi^{\bar c c}_{ab} = \frac{i\delta_{ab}}{p^2 + i\epsilon}  
\end{equation}
We draw the propagator with a dotted line, oriented by an arrow from the bar ghost to the ghost. For the charged fermion, we draw a solid line oriented by the arrow and indexed by the field $i,j$ and the spin $\alpha, \beta$. The propagator is 
\begin{equation}
    i(\Pi_{ij}^{\psi \bar \psi})_{\alpha \beta} = \left(\frac{i\delta}{\slashed{p} - m + i\epsilon}\right)
\end{equation}
Finally, the charged scalar is represented by a dashed line with long dashes, oriented by an arrow and indexed by the fields.
\begin{equation}
    i\Pi^{\phi\phi^*}_{ij} = \frac{i\delta_{ij}}{p^2 - M^2 + i\epsilon}
\end{equation}

\subsubsection{Vertices}
Gauge self interaction 
\begin{equation}
    \begin{aligned}
        \mathcal{L}^{A^3}_{\text{int}} =& -\frac{1}{4}(F_a)_{\mu\nu}(F_a)^{\mu\nu} \big|_{A^3} \\ 
        =& -g f_{abc}\partial_\mu (A_a)_\nu (A_b)^\mu (A_c)^\nu
    \end{aligned}
\end{equation}
It is a cubic vertex where each line carries a Lorentz index and a Lie algebra index. Assume all momentums are inflowing, with $k,p, q$, and recall that a $\partial_\mu$ gives in momentum space a $-ik_\mu$. To give a value to the vertex, we should add a $i$ in front (as a result of expanding $e^{iS}$) and suppose we are contracting with 3 other fields. There are 6 ($3!$) possible contractions, and we get 
\begin{equation}
    \begin{aligned}
        -g f_{abc}&(k_\nu \Pi^{\mu \kappa}_{ad}(k)\Pi^{\nu \lambda }_{be}(p)\Pi^{\rho\sigma}_{cf}(q) + \dots) \\
        &= -g \Pi^{\mu \kappa}_{ad}(k)\Pi^{\nu \lambda }_{be}(p)\Pi^{\rho\sigma}_{cf}(q)(f_{abc} k_\nu g_{\mu\rho} + f_{abc}k_\rho g_{\mu\nu} + \dots) 
    \end{aligned}
\end{equation}
such that when the dust settles, we get for the vertex 
\begin{equation}
    -g f_{abc}(g_{\mu\rho} (k_\nu - q_\nu) + g_{\mu\nu}(p_\rho - k_\rho) + g_{\nu\rho} (q_{\mu} - p_\mu))
\end{equation}
Now, let's look at the quartic term 
\begin{equation}
    \mathcal{L}^{A^3}_{\text{int}} = -\frac{1}{4}g^2 f_{eab}f_{ecd} (A_a)_\mu (A_b)_\nu (A_c)^\mu (A_d)^\nu
\end{equation}
We get for the vertex 
\begin{equation}
    \begin{aligned}
        -ig^2 [ & f_{eab}f_{ecd}(g_{\mu\rho}g_{\nu\sigma} - g_{\mu \sigma}g_{\nu \rho}) \\ 
        &+ f_{eac}f_{ebd}(g_{\mu\nu}g_{\rho\sigma} - g_{\mu\sigma}g_{\nu\rho}) \\
        &+ f_{ead}f_{ebc}(g_{\mu\nu}g_{\rho\sigma} - g_{\mu\rho}g_{\nu\sigma}) ]
    \end{aligned}
\end{equation}
Now, let's turn to the ghost-gauge boson interaction 
\begin{equation}
    \mathcal L^{ccA}_{\text{int}} = gf_{abc}\partial_\mu \bar c_a A^\mu_b c_c 
\end{equation}
which gives a factor 
\begin{equation}
    g f_{abc} p_\mu
\end{equation}
with $p$ the momentum of the outgoing ghost. For the fermion-gauge boson interaction, we find 
\begin{equation}
    \mathcal L^{\bar \psi A \psi}_{\text{int}} = g \bar \psi _i \slashed{A}_a \psi _j T^a_{ij}
\end{equation}
which gives 
\begin{equation}
    ig \gamma^\mu_{\alpha \beta}T^a_{ij}
\end{equation}
For the scalar field - gauge boson interaction, 
\begin{equation}
    \mathcal L_[\text{int}]^{\phi^2A} = -ig \partial_\mu \phi_k^* A^\mu_a \phi_j T^a_{kj} + ig A^\mu_a T^a_{ik}\phi_i^* \partial_\mu \phi_k = ig T^a_{ij}(q+k)_\mu
\end{equation}
where $k$ and $q$ are the momentum of the scalar fields flowing in the direction of their $U(1)$ current. Finally, for the last interaction, we have 
\begin{equation}
    \mathcal L^{\phi^2A^2}_{\text{int}} = g^2 (A_a)_\mu A^\mu_b T^a_{ik}T^b_{kj} \phi^*_i \phi_j
\end{equation}
such that we get 
\begin{equation}
    ig^2 g^{\mu\nu} \{T^a, T^b\}_{ij}
\end{equation}
\subsection{Taking the Feynman rules for a spin}
\subsubsection{Fermion 2-point functions} 
We have obviously a straight line, but also a diagram with a gauge loop, and a counter term. In the Feynman gauge ($\xi = 1$), we have for the diagram with a gauge loop 
\begin{equation}
    \begin{aligned}
        &\int \frac{\text{d}^4k}{(2\pi)^4} ig T^a_{jk}\frac{i}{\slashed{k} - m + i\epsilon }ig \gamma^\mu T^b_{ki} i\frac{-g_{\mu\nu}\delta^{ab}}{(p-k)^2 + i\epsilon} \\ 
        &= -g^2 \sum_a (T^a T^a)_{ji} \int \frac{\text{d}^4k}{(2\pi)^4}  \frac{\gamma_\mu (\slashed{k} + m)\gamma^\mu}{k^2 - m^2 + i\epsilon } \frac{1}{(p-k)^2 - i\epsilon} 
    \end{aligned}
\end{equation}
We notice that $\sum_a (T^a T^a)_{ji}$ is the casimir element, and is equal to $C(\pi)\delta_{ij}$. For exemple, in the defining representation of $\mathfrak{su}(n)$, $C_D = \frac{n^2 - 1}{2n}$. 

\begin{remark}
    It coincides with the QED fermion self-energy up to a group theory factor $C(\pi)\delta_{ij}$.
\end{remark}

We can try to evaluate it in dimensional regularization. We first introduce the scale $\mu$ to keep $g$ dimensionless. We replace $g$ by $g\mu^{\frac{4-d}{2}}$ thanks to dimensional analysis. We have 
\begin{equation}
    -g^2 \mu^{4-d}C(\pi)\delta_{ij} \int \frac{\text{d}^4k}{(2\pi)^4}  \frac{\gamma_\mu (\slashed{k} + m)\gamma^\mu}{k^2 - m^2 + i\epsilon } \frac{1}{(p-k)^2 - i\epsilon} 
\end{equation}
Then we can do a bit of gamma algebra: $\gamma_\mu \gamma^\mu = g^\mu_\mu = d$. So 
\begin{equation}
    \gamma_\mu \slashed{k} \gamma^\mu = k_\rho \gamma_\mu \gamma^\rho \gamma^\mu = k_\rho(2g^{\mu\rho}\gamma_\mu-\gamma^\rho \gamma^\mu\gamma_\mu) = (2-d)\slashed{k}
\end{equation}
Hence
\begin{equation}
    \gamma_\mu (\slashed{k} + m) \gamma^\mu = (2-d)\slashed{k} + dm
\end{equation}
Then we introduce Feyman parameters. Recall that we have general formulas for integrals of the form 
\begin{equation}
    \int \frac{\text{d}^d k}{(2\pi)^d} \frac{k^{2a}}{(k^2 - \triangle)^b}
\end{equation}
Using 
\begin{equation}
    \frac{1}{AB} = \int_0^1 \text{d}x \frac{1}{[A + (B-A)x]^2}
\end{equation}
Writing $A = k^2 - m^2$, $B = (p-k)^2$, 
\begin{equation}
    A + (B-A)x = (k-xp)^2 - m^2 (1-x)
\end{equation}

Next shift the integration variable $k\rightarrow k + xp$, such that our expression becomes 
\begin{equation}
    -g^2 \mu^{4-d} C(\pi)\delta_{ij} \int_0^1 \text{d}x \int\frac{\text{d}^dk}{(2\pi)^d}\frac{(2-d)(\slashed{k} + x\slashed{p}) + dm}{(k^2 - \triangle + i\epsilon)^2}
\end{equation}
with $\triangle = (m^2 - p^2x)(1-x)$. We can then evaluate the whole expression in dimensional regularization, 
\begin{equation}
    \int\frac{\text{d}^dk}{(2\pi)^d}\frac{1}{(k^2 - \triangle + i\epsilon)^2} = \frac{2}{(4\pi)^{d/2}} \frac{1}{\triangle^{2-d/2}}\Gamma(\frac{4-d}{2}) \equiv (*)
\end{equation}
such that for the 1-loop term, we get 
\begin{equation}
    -g^2 \mu^{4-d} C(\pi) \delta_{ij} \int_0^1 \text{d}x ((2-d)\slashed{p}x + dm) (*)
\end{equation}
Now, we want to extract the divergences by setting $d = 4-\epsilon$ and by expanding around $\epsilon = 0^+$. We get
\begin{equation}
    ig^2 \mu^\epsilon C(\pi) \delta_{ij} \int_0^1 \text{d}x ((2-\epsilon )\slashed{p}x- (4-\epsilon)m)\frac{1}{(4\pi)^2}(4\pi)^{\epsilon/2} \triangle^{-\epsilon/2} (\frac{2}{\epsilon} - \gamma_E + \dots)
\end{equation}
which, expanded, gives 
\begin{equation}
    ig^2 C(\pi)\delta_{ij} \frac{1}{(4\pi)^2} [\frac{2}{\epsilon}\int_0^1 \text{d}x (2\slashed{p} - 4m) + \text{ finite terms}]
\end{equation}
The divergent part must be removed by the counterterm. 

\subsection{Renormalized perturbation theory}
Strategy: Express $\mathcal L$ in terms of renormalized quantities $\rightarrow$ the difference between renormalized and bare quantities must be fixed order by order to satisfy renormalization conditions. 

\subsubsection{Possible renormalization conditions}
\begin{enumerate}
    \item $MS$: we do dimensional regularization and only substract contributions $\propto \frac{1}{\epsilon}$. 
    \item $\overline{MS}$: dimensional regularization, and substract contributions $\propto (\frac{2}{\epsilon} + \ln 4\pi e^{-\gamma_E})$
    \item Physical (QED): We introduce the electron self energy $i\Sigma(p^2)$, such that with $m_R$ the renormalized mass, the 2-point fermionic correlator takes the for, 
    \begin{equation}
        iG(\slashed{p}) = \frac{i}{\slashed{p} - m_R + \Sigma(\slashed{p})}
    \end{equation}
    We similarly introduce the photon self energy $\pi(p^2)$ and the vertex (self energy?) $\Gamma^\mu$. From these, we can impose renormalization conditions such as 
    \begin{enumerate}
        \item $\Sigma(m_p) = 0$
        \item $\Sigma'(m_p) = 0$
        \item $\Pi(0) - 0$
        \item $\Gamma^\mu(0) = \gamma^\mu$
    \end{enumerate}
    such that all of our quantities are the actual physical quantities at low energy. But this kind of conditions are not accessible in QCD, since we have asymptotic charged states. 
\end{enumerate}

\subsubsection{First encounter with a YM theory encountered in nature: QCD}
The gauge group: $SU(3)$. This means we have 8 gauge bosons, or in other words $a,b,\dots = 1,\dots, 8$. The gauge fields are called gluons. \newline
Matter (interacting strongly): quarks (fermions) \newline 
Which representations? quarks transform in the defining representation of $SU(3)$. Each quark corresponds to a triplet of fields $q_i, i = 1, 2, 3$. Terminology: each quark comes in 3 colors. We assign each index to a color. \newline 

We have 6 quarks (and their associated anti-quark). They are called flavors. They are divided into 3 generations, 
\begin{equation}
    \begin{pmatrix}
        \text{up} \\ \text{down}
    \end{pmatrix} \quad 
    \begin{pmatrix}
        \text{charm} \\ \text{strange}
    \end{pmatrix} \quad 
    \begin{pmatrix}
        \text{top} \\ \text{bottom}
    \end{pmatrix} \quad 
\end{equation}
From lighter to heavier. We will see more on the doublet structure later. They have electric charge $\begin{pmatrix}
    +2/3 \\ -1/3
\end{pmatrix}$
We will see in the next subsubsection that $\alpha_s = \frac{g_s^2}{4\pi}$ is large at low energies, and small at high energies. It is called asymptotic freedom, since it is free at very high energy. It has an important consequence: strongly interacting particles are not observed at low energy. It is called confinement. Charged particles are confined in bound states: quarks and gluons do not arise as asymptotic states. In particular, the theory has a mass gap ie no massless states in the spectrum.\par \medskip 

Intuitively, the energy in any configuration of separated strongly interacting particles is beyond particle creation threshold. Trying to pull apart a bound state will create a quark antiquark pair, creating 2 bound states. However, we have no theoretical proof of this, since perturbation breaks down. Showing that confinement occurs from first principles is an open problem: 1 million if we solve it. \par \medskip 

Bound states of quarks (hadrons) are color singlets. It means that they are in the trivial representation of $SU(3)$ (ie neutral under the strong interaction). Baryons are bound states of 3 quarks. Indeed, the tensor product of 3 fundamental representations of $SU(3)$ is the sum of a 10 dimensional representation, two 8 dimensional representations, and one 1 trivial representation. We can thus have 3 quarks belonging to a trivial representation of $SU(3)$. By the same reasonning, we see that 2 quarks bound states are impossible. On the other hand, more than 3 quarks bound states are unstable and too short lived. A baryon with 2 quarks up and one down is a proton, a baryon with 2 quarks down and one up is a neutron. We can also have mesons, which are made of a quark and an antiquark since the tensor of the fundamental and conjugate fundamental representation make a 8 dimensional representation summed with a trivial one. For exemple, we have the following pions 
\begin{equation}
    \begin{aligned}
        \frac{1}{\sqrt{2}}(u\bar{u} - d\bar{d}) &= \pi^0 \\
        \bar{u}d &= \pi^- \\ 
        u\bar{d} &= \pi^+
    \end{aligned}
\end{equation}
But if quarks do not arise as asymptotic particles, how do we measure their physical properties? Collider experiments sensitive to quarks charges, quark masses, $\alpha_s$. \par \medskip 
Let's consider $e^+ e^- \rightarrow \text{hadrons}$. This process factorizes as 
\begin{equation}
    e^+ e^- \rightarrow q\bar{q} \rightarrow \text{hadrons}
\end{equation}
We don't need the details of the hadronization when measuring the total cross-section. Electrons and Quarks are both electromagnetically and weakly charged, so we have two diagrams for the first part. We can compare it to the electron positron to muons diagrams. \par \medskip 

How are quark masses defined if not asymptotically? They are renormalization scheme dependant. For exemple, in the $\overline{MS}$ scheme, we have the masses 4.7 MeV for the down, 2.82 MeV for the up, 93.5 MeV for the strange, 1.28 GeV for the charm, 4.18 GeV for the bottom, and 163 GeV for the top.  

\subsubsection{\mathcal L = \mathcal L_{ren} + \mathcal L_{\text{counterterms}}}

Consider YM coupled to a fermion. In the action written previously, 
\begin{equation}
    \begin{aligned}
        A\rightarrow A_0 &\quad g \rightarrow g_0 \\ 
        \psi \rightarrow\psi_0 &\quad m \rightarrow m_0 \\ 
        c \rightarrow c_0 &
    \end{aligned}
\end{equation}
We introduce the renormalized charge $g_0 = Z_g g_R$, the renormalized mass $m_0 = Z_m m_R$ and the renormalized fields $A_0 = \sqrt{Z_3}A_R$, $\psi_0 = \sqrt{Z_2}\psi_R$, $c_0 = \sqrt{Z_2^c}c_R$. \par \medskip 

In term of these, $\mathcal L$ becomes 
\begin{equation}
    \begin{aligned}
        \mathcal L =& -\frac{1}{4}Z_3 (\partial_\mu A^a_{R,\nu} - \partial_\nu A^a_{R, \mu})^2 - \frac{Z_3}{2\xi}(\partial_\mu A^\mu_R)^2 \\ 
        &- Z_2 \bar \psi_R (i\slashed{\partial} - Z_m m_R)\psi_R - Z_2^c \bar{c}_R^a \square c_R \\ 
        &- Z_g Z_3^{3/2}g_R f^{abc}(\partial_\mu A^a_{R,\nu}) A^{b,\mu}_R A^{c,\nu}_R \\ 
        &-\frac{1}{4}Z^2_g Z^2_3 g^2_R (f^{eab}A^a_{R,\mu} A^b_{R,\nu})(f^{ecd} A^{c,\mu}_R A^{d,\nu}_R) \\ 
        &+ Z_g Z^{1/2}_3 Z_2 A^a_{R,\mu} \bar \psi_R \gamma^\mu T^a \psi_R + Z_g Z^{1/3}_3 Z^c_2 f^{abc}(\partial_\mu \bar c^a_R) A^{b,\mu}_R c^c_R
    \end{aligned}
\end{equation}

We introduce $Z^{A^3} = Z_g Z_3^{3/2}$, $Z^{A^4} = Z_g^2 Z_3^2$, $Z_1 = Z_gZ_3^{1/3}Z_2$, $Z^c_1 = Z_gZ_3^{1/2}Z^c_2$. The fact that the renormalization of these interaction terms can be expressed in terms of the renormalization of just the fields, the mass and one coupling is a consequence of gauge invariance. This leads to the notion of charge universality. \par \medskip 


\end{document}
