\documentclass[a4paper]{book}

% packages % 
\usepackage[utf8]{inputenc} 
\usepackage{fvextra}
\usepackage{csquotes}
\usepackage[french, italian, spanish, english]{babel}
\usepackage[T1]{fontenc}   
\usepackage{color}  
\usepackage{amsmath, dsfont, amssymb, amsthm, stmaryrd}
\usepackage[style=alphabetic]{biblatex}
\usepackage{enumitem}
\usepackage[hidelinks]{hyperref}

% graphics %
\usepackage{graphicx}
\graphicspath{ {./images/} }

% environments %
\newtheorem{theorem}{Theorem}[section]
\newtheorem{corollary}{Corollary}[theorem]
\newtheorem{lemma}[theorem]{Lemma}

\theoremstyle{definition}
\newtheorem{definition}{Definition}[section]

\theoremstyle{remark}
\newtheorem*{remark}{Remark}
\newtheorem*{example}{Example}



% bibliography %
\bibliography{bibliography} 

\begin{document}

% title %
\title{Random Geometry and Non-unitary QFT}
\author{Buisine Léo\\Ecole Normale Superieure of Paris}
\maketitle

\tableofcontents

\chapter{Introduction}
Probleme de percolation. On se place sur une grille hexagonale, a $p=\frac{1}{2}$ probabilite critique, et on prend un amas infini qui rejoint les deux bords de l'univers. Plusieurs questions 
\begin{enumerate}
    \item Quelle est la longueur du bord de l'amas? $\sim L^{7/4}$ (trouve en 1986)
    \item Quelle est la longueur du perimetre externe (si on saute par dessus les fjords)? $\sim L^{4/3}$ (trouve en 2000)
    \item Quelle est la masse de l'amas? $\sim L^{91/48}$ (trouve en 1982)
    \item Quelle est la masse de l'amas utile (le courant electrique peut passer dedans)? $\sim L ^{1,65...}$ (le nombre a une expression exacte depuis 2023, c'est un nombre transcendental)
\end{enumerate}

On cherche a generaliser le modele. On definit un etat par ses boucles (frontieres entre les cases colorees et non colorees). On peut ecrire le poid de chaque boucle $n$, et le poid de chaque trait de frontiere $K$. On a 
\begin{equation}
    Z = \sum_{\text{boucles}} n^{\# \text{boucles}} K^{\#\text{pas}}
\end{equation}
Puis on ajuste $K$ en fonction de $n$ pour etre a l'etat critique, $K(n)$. Mais on a un probleme: est ce que le modele est local? Comment l'ecrire proprement? La solution va passer par des algebres de boucles \par \medskip 

\chapter{Approche de matheux}
L'approche des mathematiciens est de se placer directement dans la limite du continu, et de considerer une courbe. Mais pour bien comprendre, regardons le modele d'Ising sur un reseau. Comment le definir? Il y a plusieurs manieres. 
\begin{enumerate}
    \item La premiere maniere est de regarder un reseau, placer un $+$ ou un $-$ sur chaque cellule, et de regarder les paroies entre les $+$ et les $-$. On a 
    \begin{equation}
        Z_{\text{Ising}} = \text{Tr}(e^{\beta J \sum_{<ij>}S_i S_j})
    \end{equation}
    avec $S_i = \pm 1$, et la trace prend la somme normalisee sur les differentes configurations. Dans ce modele, le cout d'un pas (d'une paroie) est $e^{-2\beta J}$
    \item Il existe une autre maniere de faire, dual. Cette fois, imaginons que les spins sont places sur les sommets du reseau, au lieu de les placer sur les graphes. On remarque que $e^{\beta J S_iS_j} = \text{cosh}(\beta J) + S_i S_j \text{sinh}(\beta J) = \text{cosh}(\beta J) ( 1 + xS_iS_j)$. avec $x = \text{tan}(\beta J)$. Donc 
    \begin{equation}
        Z = \text{cosh}(\beta J)^N \text{Tr} \prod_{<ij>} (1 + x S_iS_j)
    \end{equation}
    Mais si il y a $i$ tel que $S_i$ apparait un nombre impair de fois, alors prendre la trace amene la contribution du graphe a 0. Si on regarde les graphes ou on colorie les $xS_iS_j$, alors la trace nous contraint a des soments avec un nombre pair d'arretes colores. Pour un reseau avec strictement moins de 4 voisins, ca laisse seulement la possibilite de 0 ou 2 voisins. On regarde donc les boucles. Remarquons qu'on regarde les memes boucles que dans la caracterisation d'au dessus. 
    \begin{equation}
        Z_{\text{Ising}} \propto \sum_{\text{boucles}} x^{\#\text{pas}}
    \end{equation}
    De ce point de vue, a $x<x_c$: peu de boucles, les boucles sont courtes. A $x = x_c$, la longueur de la plus grande boucle diverge. A $x > x_c$, autre cas d'un system critique: on tombe dans un modele de percolation a $x=1$. 
\end{enumerate}
Ou se trouve le point critique? Evidemment quand le parametre de developpement des deux approches est le meme, quand les deux developpements produisent la meme physique de boucles. On trouve $(\beta J)_c = \frac{1}{2}\log (1+\sqrt 2)$. \par \medskip 

Fonction de correlation
\begin{equation}
    <S_{r_1} S_{r_2}> = \frac{\text{Tr}(S_{r_1} S_{r_2})\prod _{<ij>}e^{\beta J S_i S_j}}{Z}
\end{equation}
On regarde cette les fois configurations de boucles ou en plus, il y a un chemin de $r_1$ a $r_2$. \par \medskip 

Processus d'exploration\newline 
Regardons maintenant un reseau qui vit dans le demi plan $\mathbb{H}$. On fixe aussi tous les spins vivants sur le bord. On fixe l'origine, et on fixe d'un cote de l'origine les spins a $+1$ et de l'autre cote a $-1$. Enfait, en regardant la paroie partant de l'origine, on cree forcement un chemin infini auto-evitant qui ne peut jamais traverser le bord du plan. 

\begin{theorem}[Conditionnement] Soit $\gamma_1$ la partie du chemin construite apres un certain nombre de pas. Alors, la distribution de probabilite sur le restant de la courbe, conditionnee par $\gamma_1$, est egale a la distribution de probabilite pour une courbe dans $\mathcal D \ \gamma_1$, avec $\mathcal{D}$ le domaine initial de propagation du chemin.
\end{theorem}

Regardons maintenant le modele $O(n)$, definit par 
\begin{equation}
    Z_{O(n)} = \sum_{\text{boucles}} x^{\#\text{pas}} n^{\#\text{boucles}}
\end{equation}
On peut montrer que $x_c = (2 \pm \sqrt{2-n})^{-2}$. Il y a deux points critique, pour $x_c = (2 + \sqrt{2-n})^{-2}$ on obtient un point critique repulsif (sous le groupe de renormalization, a $n$ fixe), correspondant au model d'Ising pour $n=1$. De l'autre cote, pour $x_c = (2 - \sqrt{2-n})^{-2}$ on obtient un point critique attractif (sous le groupe de renormalization a $n$ fixe), correspondant a la percolation pour $n=1$. \par \medskip 
Pour $n=0$, on a pas de boucle, juste une courbe de dimension fractale. Ca correspond a un polymere. Pour $n=2$, on peut imaginer que les boucles sont orientees et on retrouve des boucles de poid 1 (vu que deux orientations possibles donnent un poid de 2). Dans ce cas, on peut imaginer que les boucles correspondent a une carte topographique, ou l'orientation signifie que le niveau descend a droite quand on va dans le sens de la fleche. On obtient donc un relief. En prenant la limite du continu et en supposant que le niveau varie de moins en moins plus la taille du systeme grandit, on obtient un relief montagneux aleatoire = un champ gaussien libre du point de vue de matheux (pour nous physiciens c'est un champs bosonique libre). Un autre cas interessant (mais ca tue les matheux) c'est le cas $n=-2$. Si on prend cette theorie dans le continue et qu'on regarde les fonctions de green, on se rend compte que notre theorie correspond a une loop-erased random walk. (une marche aleatoire mais ou on efface qd on fait une boucle). C'est aussi equivalent a un tas de sable abelien. (on a une assiette et on peut empiler des grains sur chaque site, mais qd on a une pile qui depasse de 4 (nb de voisins du site) sur un site ca se repartit sur les sites autours). \par \medskip 

Autre generalisation du modele d'Ising. \newline 
Cette fois, on prend un reseau carre, et on prend 
\begin{equation}
    Z_{\text{Potts}} = \text{Tr}(e^{\beta J \sum_{<ij>}\delta_{\sigma_i \sigma_j}})
\end{equation}
avec $\sigma_i = 1,\dots , Q$. Pour Q = 2, on a Ising. En remarquant que 
\begin{equation}
    e^{\beta J \delta _{\sigma_i \sigma_j}} = 1 + (e^{\beta J} - 1)\delta _{\sigma_i \sigma_j}
\end{equation}
On peut regarder les graphes ou on trace une arrete si on place $\delta _{\sigma_i \sigma_j}$ sur ce site. On obtient 
\begin{equation}
    Z = \sum_{A \text{ graph}} (e^{\beta J} - 1)^{|A|} Q^{K(A)}
\end{equation}
ou $K(A)$ est le nombre de de composantes connexes de $A$. 
On remarque ici qu'on peut oublier que $Q$ doit etre entier, et prendre n'importe quel $Q$ (please take $Q$ positif reel diront les matheux). $Q \rightarrow 1$ donne un modele de percolation d'arretes avec $e^{\beta J} - 1 = \frac{p}{1-p}$, $p\in [0,1]$. $Q = 2$ donne le modele d'Ising, n'importe quel $Q > 2$ donne un modele interessant, mais seulement $0\leq Q \leq 4$ donnera un point critique (de deuxieme ordre).\par \medskip
De meme que la dualite entre haute et basse temperature du modele d'Ising, on a une dualite sur le modele de Potts en regardant des graphes sur le dual du reseau. Encore une fois, on se rend compte que le point critique est quand les duaux on le meme parametre, ce qui donne 
\begin{equation}
    e^{(\beta J)_c} - 1 = \sqrt{Q}
\end{equation}
On retrouve (mais par pure coincidence, c'est pas le meme reseau) que pour Ising on a $(\beta J)_c = \ln (1 + \sqrt{2})$. Pour $Q \rightarrow 0$, on trouve des UST (uniform spanning tree). 

\chapter{Stochastic Loewner evolution}

$SLE_\kappa$ ( with parameter $\kappa$).\newline  
See https://en.wikipedia.org/wiki/Schramm\%E2\%80\%93Loewner\_evolution \newline
On cherche a aplatir la courbe avec $g_t(z)$ (une conformal map), qui obeit une equation differentielle avec $a_t = g_t(\gamma(t))$ est l'endroit ou est envoye le bout de la corde. 
\begin{equation} 
    \frac{\text{d}g_t(z)}{\text{d}t} = \frac{z}{g_t(z) - a_t}
\end{equation}
2 proprietes: le conditionnement et le transport conforme. 

\begin{theorem}[Schramm] Si on a les 2 proprietes, plus la symetrie de reflexion, alors 
    \begin{equation}
        a_t = \sqrt{\kappa} B_t
    \end{equation}
\end{theorem}
$a_t$ est proportionel a un mouvement brownien, avec $\kappa \geq 0$ le seul parametre qu'on peut varier. 
\begin{remark}
    \begin{equation}
        <a_t> = 0
    \end{equation}
    \begin{equation}
        <(a_{t_1} - a_{t_2})^2> = \kappa |t_1-t_2|
    \end{equation}
    $\kappa$ est relie a la charge centrale de la theorie conforme. 
\end{remark}

\section{Les phases de SLEk}
On definit 
\begin{equation}
    x_t = g_t(x_0) - a_t
\end{equation}
Ou autrement dit, de maniere moins belle
\begin{equation}
    \text{d}x_t = \frac{2\text{d}t}{x_t} - \sqrt{\kappa}\text{d}B_t
\end{equation}
(ce qui en physique, donne le processus de Bessel avec equation de Langevin $\dot x = \frac{2}{x} - \eta_t$ avec $\eta_t$ un bruit blanc). 
On voit que $x_t$ se fait expulser de l'origine. Qui gagne, entre le potentiel qui eloigne de l'origine et la force stochastique? who winnns \par \medskip 

Sans la force aleatoire, on aurait 
\begin{equation}
    x_t^2 = 4t
\end{equation}
Alors que sans le terme repulsif, on aurait
\begin{equation}
    <x_t^2> = \kappa t 
\end{equation}
Donc le terme repulsif emporte ssi $\kappa < 4$. Dans ce cas, on peut aussi s'attendre a ce que la particule s'echappe a l'infini, qu'elle ne retourne pas sur ses pas et que la courbe soit simple. On s'attend donc aussi a ce qu'elle ne touche pas l'axe reel. Si par contre $\kappa > 4$, le bruit l'emporte. La particule colisionne avec 0 en temps fini et on a une singularite de $g_t$, vu que toute l'air disparait d'un coup. \par \medskip 

On peut meme aller plus loin et calculer la dimension fractale de la courbe: 
\begin{equation}
    \begin{aligned}
        d_f &= 1 + \frac{\kappa}{8} \text{ si } \kappa < 8 \\
        &= 2 \text{ sinon}
    \end{aligned}
\end{equation}
Pour pile $\kappa = 8$, on a un modele. C'est le modele de Potts
\begin{equation}
    Z = \sum_{A \subset E} (e^{K-1})^{|A|} Q^{K(A)}
\end{equation}
dans la limite $Q \rightarrow 0$. Points autodual sur le graph: $e^{K_c} - 1 = \sqrt{Q}$. \par \medskip 

On a aussi une dualite entre courbes avec et sans point doubles: si on a une courbe pour $\kappa > 4$, on peut tracer son contours, qui donc n'aura pas de point double, et correspondra a une courbe avec $\tilde \kappa = \frac{16}{\kappa} < 4$. On remarque donc que $\kappa = 4$ est son autodual. C'est la dualite de Duplantier. On remarquera d'ailleurs que $\kappa$ et $\tilde \kappa$ decrivent la meme CFT avec la meme charge centrale, mais avec des observables differents. 
\begin{equation}
    c = 1 - \frac{6(g-1)^2}{g} \qquad g = \frac{4}{\kappa}
\end{equation}
Dans le discret, c'est equivalent a un modele de boucles avec $n = -2\text{cos}(\pi g)$
\begin{equation}
    Z_{O(n)} = \sum_{\text{boucles}} K^{\#\text{pas}} n^{\#\text{boucles}}
\end{equation}
On a aussi les equivalences 
\begin{enumerate}
    \item $\kappa = 2, n = -2$: Loop erased random walk
    \item $\kappa = \frac{8}{3}, n = 0$: Polymeres (de Gennes)
    \item $\kappa = 4, n = 2$: modele de hauteurs (champs libre Gaussien)
    \item $\kappa = 6, n = 1$: Percolation (conformal invariance prouvee)
    \item $\kappa = 8, n = 0$: Polymere avec $d_f = 2$
\end{enumerate}

Un exemple de calcul avec $SLE_k$: on place un point $\zeta$ dans le demi plan, et on pose la question de si $\gamma$ se situe a la gauche de $\zeta$? La probabilite depend de $\zeta$ et $\bar \zeta$, et de $a_0$. Evidemment, la technique est comme toujours de regarder l'evolution. On ecrit que la probabilite d'avoir une courbe a gauche en partant de $a_0$ est la meme que la moyenne de probabilite en partant de $a_0 + \sqrt{\kappa} B_t$ avec $B_t$ un element stochastique. On developpe a l'ordre $\text{d}t$. On obtient la formule de Schrawn pour $\kappa = 6$. \par \medskip 

\chapter{Methodes algebriques}

\section{L'algebre de Temperley-Lieb et modele de Potts}
Un graph est decrit par $G = (V, E)$. On prend
\begin{equation}
    Z_\text{Potts} = \sum_{A \subset E} v^{|A|}Q^{k(A)}
\end{equation}
En TD, on a vu $k(A) = |V| - |A| + c(A)$. avec $c(A)$ les cycles de A, $|V|$ le nombre de sites du reseau, $k(A)$ le nombre de composantes connexes (un point [cad un site non couvert par un pas] est considere comme composante connexe, donc chaque point non couvert contribue a hauteur de $Q$), $|A|$ le nombre de pas. \par \medskip 

Donne un graphe, on peut dessiner les boucles formees par le dessin. Le nombre de boucles $l(A)$ est donne par $l(A) = k(A) + c(A)$. On definit $x = \frac{v}{\sqrt{Q}}$. On a 
\begin{equation}
    Z_{\text{Potts}} = Q^{|V|/2}\sum_{A\subset E} x^{|A|} Q^{l(A)/2}
\end{equation}
Prenons $G$ un reseau carre. On a une dualite pour $x_c = 1$ (cf TD). A une constante globale pres a $x_c = 1$, on a 
\begin{equation}
    Z = \sum_{A\subset E} \beta^{l(A)} \quad \text{avec  } \beta = \sqrt{Q}
\end{equation}
Les regles pour tracer une boucles sont qu'a chaque lien entre 2 sites (vertex potentiel) on dessine un X (><) dont le sens depend de si le lien est occupe par le graph ou pas. En gros, on traverse les liens pas occupes et on ne traverse pas les liens occupes. 

\section{The Temperley Lieb algebra}

We have $TL_n$, generators $1, u_1, u_2, \dots, u_{n-1}$. 
\begin{equation}
    \begin{aligned}
        &u_i ^2 = \beta u_i \\
        &u_i u_{i+1} u_i = u_i \\ 
        &u_i u_j = u_j u_i \quad \text{for } |i-j| > 1
    \end{aligned}
\label{eq:laws}\end{equation}
where in terms of diagrams, $u_i$ is straight line everywhere except for between $i$ and $i+1$ where we have $\subset$ on both sides. \par \medskip 

Imagine we have a large loop in 2D. Given two points, the probability that they belong to the same loop may be equal to 
\begin{equation}
    \mathbb{P}(x, y) = \frac{1}{|x-y|^{2\triangle}}
\end{equation}
Then the fractal dimension of the loop is $d_f = 2-\triangle$. \par \medskip 

We look at $(n,p)$ link states, with $n$ points, $p$ pairings, $n-2p$ defect lines. Theses are in bijections with walk on $\mathbb{Z}$ from $(0, 0)$ to $(n-p,p)$ staying below the diagonal. We need to do the same for words made out of $u_i$. 

\begin{lemma}
    Let $U$ be a word made out of $u_i$, reduced by the laws \eqref{eq:laws} to some word as short as possible, then the max index $m$ only occurs once. 
\end{lemma}
Let $U$ be reduced maximally. We can write 
\begin{equation}
    U = U' . (u_m u_{m-1} u_{m-2} \dots u_{m-l})
\end{equation}
where $U'$ has max index lower than $m$, and the other term is a descending tail of length $l$. \par \medskip 

Jones normal form: 
\begin{equation}
    U = (u_{j_1} u_{j_1 - 1} \dots u_{k_1}) (\dots ) (u_{j_\sigma} u_{j_\sigma - 1}\dots u_{k_\sigma})
\end{equation}
where comparing to the lemma above, $j_\sigma = m$, $k_\sigma = m-l$, and to make it a normal form: 
\begin{equation}
    \begin{aligned}
        &0 < j_1 < j_2 < \dots < j_\sigma < n \\ 
        &0 < k_1 < k_2 < \dots < k_\sigma < n \\ 
    \end{aligned}
\end{equation}
Moreover, we can prove that 
\begin{equation}
    j_i \geq k_i
\end{equation}
The number of such possibilities is the Catalan number. Another way to see this is that the number of $(n,p)$ link states is 
\begin{equation}
    d_{n,p} = \binom{n}{p} - \binom{n}{p-1}
\end{equation}
In particular, 
\begin{equation}
    \dim TL_n = d_{2n,n} = \frac{1}{n+1}\binom{2n}{n} = C_n
\end{equation}

Standard representation: An $A$-module $(M,+,.)$ is a set with $+$ such that $(M,+)$ is an abelian group, and such that the dot is an action from $A$ to $M$. We will look at modules of the temperley lieb algebra. 

We make $TL_n$ act on the links, by the most natural left action. Not that the number of pairings can only increase under the action. 

\begin{definition}[Link Modules]
    Let $M_n$ be the complex span of all $(n,p)$ link states. Then $M_n$ is a left-$TL_n$ module. 
\end{definition}
\begin{definition}
    Let $M_{n,p}$ be $(n,p')$ links with $p'\geq p$. $M_{n,p}$ is stable under the left $TL_n$ action. 
\end{definition}
In particular, we have the following filtration 
\begin{equation}
    0 \subset M_{n, [\frac{n}{2}]} \subset \dots \subset M_{n,1} \subset M_n
\end{equation}
We define 
\begin{equation}
    V_{n,p} = \frac{M_{n,p}}{M_{n,p+1}}
\end{equation}
\end{document}