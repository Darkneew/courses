\documentclass[a4paper]{book}

% packages % 
\usepackage[utf8]{inputenc} 
\usepackage{fvextra}
\usepackage{csquotes}
\usepackage[french, italian, spanish, english]{babel}
\usepackage[T1]{fontenc}   
\usepackage{color}  
\usepackage{amsmath, dsfont, amssymb, amsthm, stmaryrd}
\usepackage[style=alphabetic]{biblatex}
\usepackage{enumitem}
\usepackage[hidelinks]{hyperref}

% graphics %
\usepackage{graphicx}
\graphicspath{ {./images/} }

% environments %
\newtheorem{theorem}{Theorem}[section]
\newtheorem{corollary}{Corollary}[theorem]
\newtheorem{lemma}[theorem]{Lemma}

\theoremstyle{definition}
\newtheorem{definition}{Definition}[section]

\theoremstyle{remark}
\newtheorem*{remark}{Remark}
\newtheorem*{example}{Example}



% bibliography %
\bibliography{bibliography} 

\begin{document}

% title %
\title{Seminars}
\author{Buisine Léo\\Ecole Normale Superieure of Paris}
\maketitle

\tableofcontents

\chapter{Variations sur le groupe de Galois cosmique}
Alain Connes, Pierre Cartier \par \medskip 

Il s'est tout de suite accroche aux algebres de Hopf de diagrammes de Feynman. Idee venue avec Kreimer en discutant de la renormalization: comment ca marche? 

\section{Groupe de Galois cosmique}
Renormalisation et ambiguite galoisienne. \par \medskip 

Calculer des fonctions de correlations necessite un developpement perturbatif, ce qui donne des integrales divergentes indexees par des graphes de Feynman. La technique utilisee le plus couramment et la regularization dimensionelle. \par \medskip 

https://encyclopediaofmath.org/wiki/Birkhoff\_factorization \par \medskip 

Noncommutative Geometry, Quantum Fields and Motives


\section{Corps de constantes en physique}
Anneaux de Fontaine a la place archimedienne \par \medskip 

Geometrie algebrique, tropicale, theorie des nombres, dequantization

Reecrire les lois de thermodynamiques en utilisant a la place de la somme la convolution $(f\star g) (z)= \sup_{x+y = z} f(x)g(y)$. En particulier la transformee de Legendre est dans ce langage celle de Fourier.

la quantification, c'est la construction de Witt



\end{document}