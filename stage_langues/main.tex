\documentclass[a4paper]{report}

% packages % 
\usepackage[utf8]{inputenc} 
\usepackage{fvextra}
\usepackage{csquotes}
\usepackage[french, italian, spanish, english]{babel}
\usepackage[T1]{fontenc}   
\usepackage{color}  
\usepackage{amsmath, dsfont, amssymb, amsthm, stmaryrd}
\usepackage[style=alphabetic]{biblatex}
\usepackage{enumitem}
\usepackage[hidelinks]{hyperref}

% graphics %
\usepackage{graphicx}
\graphicspath{ {../img/} }

% environments %
\newtheorem{theorem}{Theorem}[section]
\newtheorem{corollary}{Corollary}[theorem]
\newtheorem{lemma}[theorem]{Lemma}

\theoremstyle{definition}
\newtheorem{definition}{Definition}[section]

\theoremstyle{remark}
\newtheorem*{remark}{Remark}

% bibliography %
\bibliography{bibliography} 

\begin{document}

% title %
\title{Stage Intensif de Langue}
\author{Buisine Léo\\Ecole Normale Superieure of Paris}
\maketitle

\tableofcontents

\chapter{Ancien Français}

C'est la matière la plus redoutée: gare à nous! \par

C'est un stage intensif: mise à niveau? Mais quel niveau? Orienté pour les débutants, ceux n'ayant jamais fait d'ancien français, ceux issus de CPGE ou ceux qui ont peur de l'ancien français. \par

On va travailler sur l'oeuvre au programme, \textit{Le Bestiaire D'Amour et La Reponse Du Bestiaire}.

\section{Phonetique historique}

Cette matière est aride au départ. Demande de maitriser beaucoup de fondamentaux en peu de temps, mais après ça devient ludique. C'est ça qui tient tout l'ancien français, raison pour laquelle on commence par là. \par

\subsubsection{Alphabet Bourciez}

Pour l'alphabet, le plus usité est l'API. MAIS les gens chiants d'ici utilisent l'alphabet romaniste ou Bourciez, donc on va utiliser lui. Comment l'utiliser? Phonétique = on s'intéresse à la prononciation. \par 

Un phone est transcrit entre "[]". Si il y a un point souscrit sous la voyelle entre crochés droit, le son est fermé. La même chose avec un demi "c" en dessous signifie que le son est ouvert. Il existe deux types de a, le "[a]" de "patte" et le "[â]" qui est un a vélaire, comme dans "pâte". On met un "~" au dessus d'une voyelle nasale. Le "oe" velaire avec un c dessous correspond à "brun", tandis que le o velaire avec un c dessous correspond a "brin". C'est plus étiré? Un "e" avec un rond dessous correspond au e central (e muet). En réalité en ancien français c'est pas muet du tout. Le "l" avec un v en dessous est le ll en espagnol. Le n avec un v en dessous est le n accent ~ en espagnol. Le "l" de cheval articulé est un l vélaire, noté l barré. Mettre un accent v au dessus d'une consonne veut dire qu'il est ?, donc s accent v se dit ch, et le z accent v se dit j. Un $\beta$ correspond a un v espagnol.\par 

Consonne sonore = cordes vocales vibrent. Consonne sourde = corde vibre pas. Consonne occlusive = impossible a prolonger. Consonne constrictive = prolongeable \par

Les voyelles peuvent s'articuler dans plusieurs zones: elles peuvent venir de l'avant (voyelle avant/antérieure/palatale) ou elles peuvent venir de l'arrière (arrière/postérieure/vélaire). Geographiquement, on place l'avant a gauche et l'arrière a droite. De plus, on classe le plus fermé en haut et le plus ouvert a droite. On peut distinguer les voyelles en deux séries. La première des voyelles avant étirées (on étire la bouche pour la prononcer): $[i], [e], [\underset{c}{e}], [a]$. La deuxième série correspond aux voyelles avant arrondies (labiales, on arrondi les lèvres pour les articuler): $ü, oe, \underset{c}{oe}$. La dernière série est faite des voyelles qui sont arrières labiales: $u, \underset{.}{o}, \underset{c}{o}, â$. Le e central est entre la deuxième et la troisième série, au milieu.


\subsubsection{Les syllabes}

La première sylablle d'un mot est l'initiale. La prétonique est la syllable avant le ton: prétonique interne ou externe en fonction de si elle est dans ou au bord du mot. On parle de finale, pénultieme, antepenultieme en partant de la fin. En français, on place l'accent sur la dernière voyelle non muette. \par

Gouvernail : GUBERNACULUM. L'accent en français est sur le a de ail. L'accent ne change pas de place entre l'ancien français et le français moderne. La syllable accentué était donc le "NA" en latin, et la syllabe prétonique est interne: "BER".\par

Il est facile de distinguer les syllabes. Mais méfions nous des diphtongues et des hiatus. Une diphtongue est deux voyelles en un son. Il en existait 3 en latin: "ae", "oe" (ouais), "au". Il n'en existe plus en français moderne. La plupart des diphtongues meurent au 13e, et le reste au 16e.\par

Un hiatus: deux syllabes cote a cote qui ne sont pas un diphtongue. Dans un diphtongue, les deux voyelles font une seule syllabe. Dans le cas du hiatus, il y a deux syllabes différentes. Ex: filia = fi.li.a. AU milieu du hiatus, il y a une dièrèse.\par

Syllabe fermée = entravée: termine par une consonne. Syllabe ouverte = libre: termine par une voyelle.

\subsection{Calcul des quantites des voyelles latines}

Les grandes règles numérotées qui nous ont été données par Dieu (la prof)

\begin{enumerate}
    \item REGLE NUMERO 1: L'accent est toujours sur la dernière voyelle non muette
    \item REGLE NUMERO 2: L'accent ne change pas de place entre le latin et le français
\end{enumerate} \par

L'accent en latin ne peut jamais être sur la dernière syllabe (sauf monosyllabe). Une syllabe accentuée ne peut jamais s'amenuir du latin au français. \par

GAUDIAM > joie \par

Comment trouver l'accent? "joie" est monosyllabique, ca peut etre soit "GAU" soit "DI" (pas d'accent sur la syllabe finale). Enfait, il y avait des voyelles longues et d'autres courtes en latin qui déterminait l'accent, système remplacé par des voyelles plus ou moins fermées en français moderne.

Suite des règles de Dieu
\begin{enumerate}
    \item REGLE NUMERO 3: L'accent n'est jamais sur la dernière syllabe en latin
    \item REGLE NUMERO 4: Monosyllable: accent sur la seule syllabe, bisyllabe: accent sur la première syllabe, sur un mot latin d'au moins trois syllabes: accent sur la pénultième si et seulement si cette syllabe est de quantité longue. Sinon, accent sur l'antepenultieme syllabe.
    \item REGLE NUMERO 5: Sur un mot latin d'au moins 3 syllabes, si la syllabe penultieme est ouverte, la quantité de la syllabe est la même que celle de sa voyelle.
\end{enumerate}

Qu'est ce qu'une quantité wtf? Quantité à apprendre par coeur: LA PREMIERE VOYELLE D'UN HIATUS EST TOUJOURS DE QUANTITE BREVE. A L'INVERSE, UNE DIPHTONGUE SERA TOUJOURS DE QUANTITE LONGUE. Dans l'alphabet Bourciez, on met une barre sur une syllabe longue (ex: $[\bar{ae}], [\bar{i}]$) et un u sur une syllabe breve (ex: $[\overset{u}{i}]$). \par

Donc dans GAUDIAM, GAU est accentué car DI est bref. \par

Seul un i long latin se maintient avec son timbre i en français. De même, seul un [u] long latin peut donner en français le son [ü]. \par 

Un oxyton est un mot accentué sur sa dernière syllabe (que monosyllabe en latin). Un paroxyton est un mot dont la penultieme syllabe est accentuée. Finalement, un mot dont l'antepenultieme syllabe est accentuée est un proparoxyton. 

Au 4eme siècle apres JC tout syllabe e/o initiale atone a un timbre fermé, quelque soit sa quantité latine.

\subsection{Diphtonguaisons spontanées}

\subsection{Nasalisation des voyelles simples}

\subsection{Les voyelles atones}

\subsection{Aperçu sur les palatalisations}

\section{morphologie}

\subsection{morphologie nominale}

\subsection{variantes combinatoires}

\subsection{morphologie verbale}

\subsubsection{futur}

\subsubsection{conditionel}

\subsubsection{imparfait}

\subsubsection{passe simple}

\subsubsection{present}

\subsection{diachronie}

\section{syntaxe}


\printbibliography

\end{document}