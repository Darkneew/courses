\documentclass[a4paper]{report}

% packages % 
\usepackage[utf8]{inputenc} 
\usepackage{fvextra}
\usepackage{csquotes}
\usepackage[french, italian, spanish, english]{babel}
\usepackage[T1]{fontenc}   
\usepackage{color}  
\usepackage{amsmath, dsfont, amssymb, amsthm, stmaryrd}
\usepackage[style=alphabetic]{biblatex}
\usepackage{enumitem}
\usepackage[hidelinks]{hyperref}

% graphics %
\usepackage{graphicx}
\graphicspath{ {../img/} }

% environments %
\newtheorem{theorem}{Theorem}[section]
\newtheorem{corollary}{Corollary}[theorem]
\newtheorem{lemma}[theorem]{Lemma}

\theoremstyle{definition}
\newtheorem{definition}{Definition}[section]

\theoremstyle{remark}
\newtheorem*{remark}{Remark}

% bibliography %
\bibliography{bibliography} 

\begin{document}

% title %
\title{Stage Intensif de Langue}
\author{Buisine Léo\\Ecole Normale Superieure of Paris}
\maketitle

\tableofcontents

\chapter{Ancien Français}

C'est la matière la plus redoutée: gare à nous! \par

C'est un stage intensif: mise à niveau? Mais quel niveau? Orienté pour les débutants, ceux n'ayant jamais fait d'ancien français, ceux issus de CPGE ou ceux qui ont peur de l'ancien français. \par

On va travailler sur l'oeuvre au programme, \textit{Le Bestiaire D'Amour et La Reponse Du Bestiaire}.

\section{Phonetique historique}

Cette matière est aride au départ. Demande de maitriser beaucoup de fondamentaux en peu de temps, mais après ça devient ludique. C'est ça qui tient tout l'ancien français, raison pour laquelle on commence par là. \par

\subsubsection{Alphabet Bourciez}

Pour l'alphabet, le plus usité est l'API. MAIS les gens chiants d'ici utilisent l'alphabet romaniste ou Bourciez, donc on va utiliser lui. Comment l'utiliser? Phonétique = on s'intéresse à la prononciation. \par 

Un phone est transcrit entre "[]". Si il y a un point souscrit sous la voyelle entre crochés droit, le son est fermé. La même chose avec un demi "c" en dessous signifie que le son est ouvert. Il existe deux types de a, le "[a]" de "patte" et le "[â]" qui est un a vélaire, comme dans "pâte". On met un "~" au dessus d'une voyelle nasale. Le "oe" velaire avec un c dessous correspond à "brun", tandis que le o velaire avec un c dessous correspond a "brin". C'est plus étiré? Un "e" avec un rond dessous correspond au e central (e muet). En réalité en ancien français c'est pas muet du tout. Le "l" avec un v en dessous est le ll en espagnol. Le n avec un v en dessous est le n accent ~ en espagnol. Le "l" de cheval articulé est un l vélaire, noté l barré. Mettre un accent v au dessus d'une consonne veut dire qu'il est ?, donc s accent v se dit ch, et le z accent v se dit j. Un $\beta$ correspond a un v espagnol.\par 

Consonne sonore = cordes vocales vibrent. Consonne sourde = corde vibre pas. Consonne occlusive = impossible a prolonger. Consonne constrictive = prolongeable \par

Les voyelles peuvent s'articuler dans plusieurs zones: elles peuvent venir de l'avant (voyelle avant/antérieure/palatale) ou elles peuvent venir de l'arrière (arrière/postérieure/vélaire). Geographiquement, on place l'avant a gauche et l'arrière a droite. De plus, on classe le plus fermé en haut et le plus ouvert a droite. On peut distinguer les voyelles en deux séries. La première des voyelles avant étirées (on étire la bouche pour la prononcer): $[i], [e], [\underset{c}{e}], [a]$. La deuxième série correspond aux voyelles avant arrondies (labiales, on arrondi les lèvres pour les articuler): $ü, oe, \underset{c}{oe}$. La dernière série est faite des voyelles qui sont arrières labiales: $u, \underset{.}{o}, \underset{c}{o}, â$. Le e central est entre la deuxième et la troisième série, au milieu.


\subsubsection{Les syllabes}

La première sylablle d'un mot est l'initiale. La prétonique est la syllable avant le ton: prétonique interne ou externe en fonction de si elle est dans ou au bord du mot. On parle de finale, pénultieme, antepenultieme en partant de la fin. En français, on place l'accent sur la dernière voyelle non muette. \par

Gouvernail : GUBERNACULUM. L'accent en français est sur le a de ail. L'accent ne change pas de place entre l'ancien français et le français moderne. La syllable accentué était donc le "NA" en latin, et la syllabe prétonique est interne: "BER".\par

Il est facile de distinguer les syllabes. Mais méfions nous des diphtongues et des hiatus. Une diphtongue est deux voyelles en un son. Il en existait 3 en latin: "ae", "oe" (ouais), "au". Il n'en existe plus en français moderne. La plupart des diphtongues meurent au 13e, et le reste au 16e.\par

Un hiatus: deux syllabes cote a cote qui ne sont pas un diphtongue. Dans un diphtongue, les deux voyelles font une seule syllabe. Dans le cas du hiatus, il y a deux syllabes différentes. Ex: filia = fi.li.a. AU milieu du hiatus, il y a une dièrèse.\par

Syllabe fermée = entravée: termine par une consonne. Syllabe ouverte = libre: termine par une voyelle.

\subsection{Calcul des quantites des voyelles latines}

Les grandes règles numérotées qui nous ont été données par Dieu (la prof)

\begin{enumerate}
    \item REGLE NUMERO 1: L'accent est toujours sur la dernière voyelle non muette
    \item REGLE NUMERO 2: L'accent ne change pas de place entre le latin et le français
\end{enumerate} \par

L'accent en latin ne peut jamais être sur la dernière syllabe (sauf monosyllabe). Une syllabe accentuée ne peut jamais s'amenuir du latin au français. \par

GAUDIAM > joie \par

Comment trouver l'accent? "joie" est monosyllabique, ca peut etre soit "GAU" soit "DI" (pas d'accent sur la syllabe finale). Enfait, il y avait des voyelles longues et d'autres courtes en latin qui déterminait l'accent, système remplacé par des voyelles plus ou moins fermées en français moderne.

Suite des règles de Dieu
\begin{enumerate}
    \item REGLE NUMERO 3: L'accent n'est jamais sur la dernière syllabe en latin
    \item REGLE NUMERO 4: Monosyllable: accent sur la seule syllabe, bisyllabe: accent sur la première syllabe, sur un mot latin d'au moins trois syllabes: accent sur la pénultième si et seulement si cette syllabe est de quantité longue. Sinon, accent sur l'antepenultieme syllabe.
    \item REGLE NUMERO 5: Sur un mot latin d'au moins 3 syllabes, si la syllabe penultieme est ouverte, la quantité de la syllabe est la même que celle de sa voyelle.
\end{enumerate}

Qu'est ce qu'une quantité wtf? Quantité à apprendre par coeur: LA PREMIERE VOYELLE D'UN HIATUS EST TOUJOURS DE QUANTITE BREVE. A L'INVERSE, UNE DIPHTONGUE SERA TOUJOURS DE QUANTITE LONGUE. Dans l'alphabet Bourciez, on met une barre sur une syllabe longue (ex: $[\bar{ae}], [\bar{i}]$) et un u sur une syllabe breve (ex: $[\overset{u}{i}]$). \par

Donc dans GAUDIAM, GAU est accentué car DI est bref. \par

Seul un i long latin se maintient avec son timbre i en français. De même, seul un [u] long latin peut donner en français le son [ü]. \par 

Un oxyton est un mot accentué sur sa dernière syllabe (que monosyllabe en latin). Un paroxyton est un mot dont la penultieme syllabe est accentuée. Finalement, un mot dont l'antepenultieme syllabe est accentuée est un proparoxyton. 

Au 4eme siècle apres JC tout syllabe e/o initiale atone a un timbre fermé, quelque soit sa quantité latine.

\chapter{Linguistique}

Grammaire scientifique

\section{Elements de morphologie}

\subsection{Introduction}

Degrés de grammarité? Différentes echelles d'unité linguisitique \par

Phonème > Morphème > Mots > Groupes et phrase \par
Phonologie > Morphologie > Syntaxe
\begin{enumerate}
    \item Phonème : son
    \item Morphème: unité de sens 
\end{enumerate}

Bi articulation du langage: les langues sont articulées a deux niveaux de complexité:\par
d'abord a chaque mot/phrase est associé une forme et un sens\par
mais a un niveau plus bas une langue est une regle qui a des phonemes associe des morphemes 

Les disciplines 
\begin{enumerate}
    \item Phonologie: l'etude des sons pertinents dans une langue donnée (phonétique = science des sons, phonologie = science des sons pertinents)
    \item Syntaxe: l'etude de la disposition des mots dans une phrase
    \item Morphologie: l'etude de la formation des mots par des unites minimales associants forme et sens
    \item Semantique: l'etude du sens des mots et des phrases
    \item Pragmatique: l'etude du sens dans la production et la reception d'enoncés
\end{enumerate}

Mot: son ou groupe de sons articulés constituant une unité porteuse de signification dans une langue donnée; les mots possèdent une catégorie grammaticale et se combinent entre eux pour former des phrases. Ils sont classifiés en 9 catégories traditionnellement, mais 9 catégories sujets a changement. \par 

Le mot mot est imprécis: compte pas les occurences du meme mot, des différentes morphologies d'un mot, pas toujours une unité minimale, pas de distinction entre unité lexicale et grammaticale, des sequences de mots se comportent comme un seul mot (cochon d'Inde), etc\dots \par

En linguistique, on étudie le morphème. Plus petite unité linguistique dotée d'une forme et un sens. Morphème est indécomposable en plus petites unités possédants forme et sens. Morphème se combinent en mots (des fois mot = morphème, ex: ami, loutre) \par

On classifie les morphèmes en deux catégories: morphème lexicaux (tirés des sens, de la pensée) aussi appelés lexèmes, et morphèmes grammaticaux (tirés uniquement de la langue) aussi grammème. \par 

On peut classifier les morphèmes. Soit un morphème est libre (c'est un mot) soit il est lié. Si il est lié, soit c'est la base (libérable ou non) soit c'est un affixe. En affixe, soit c'est un grammème (toujours suffixe) soit c'est un lexème (soit préfixe soit suffixe). \par

Principe de segmentation commutation : principe du jackpot, chaque roue peut tourner et se faire remplacer par un aurte phonème. 

\subsection{L'allomorphie}

Quand un morphème peut avoir des formes légèrement différentes (un allomorphe). On appelle les différentes formes d'un morphème les morphes. \par

infaisable, irresponsable, illisible \par 

vapor-iser, am-abilité, trac-abil-ité

\chapter{Stylistique}

Orienté agregatifs. Cours organisés par type de texte: récit, théâtre, poésie,\dots

\section{Vocabulaire}

Bally: inventeur de la stylistique, apprenti de la linguistique. Ce n'est pas une branche de théorie littéraire, mais de linguistique. La stylistique est vague, c'est l'étude du style. Il y a de nombreuses branches et tendances en fonction de notre définition du style et de notre vision. \par 

Voici quelques branches principales
\begin{enumerate}
    \item Chaque auteur a son style, la stylistique vise a montrer l'originalité du style de l'auteur. C'est une stylistique de l'écart, cherche à appréhender le style d'un auteur et ses différences, de manière formelle. En France, l'originalité d'un auteur est souvent jaugée par sa forme, son style. Pour dire qu'un auteur est mauvais, on dit qu'il écrit mal. Un auteur est jugé sur son style. Le style légitime la qualité littéraire d'un ouvrage. En une expression ramassée, on peut essayer de résumer le style d'un auteur (l'art de la sourdine pour Racine par exemple, ou celui de la transition pour La Fontaine)
    \item Dans les exercices académiques, on ne vise pas à identifier le style d'un auteur mais à mobiliser les sciences du langage (linguisitique) comme magasin d'outils et de méthodes pour appréhender le sens d'un texte. Plus aucun jugement, plus aucune valeure donnée, on cherche juste à interpréter le texte. Dans cette vision, c'est un magasin d'outils.
\end{enumerate}

En stylistique, l'intépretation est fondamentale. En stylistique, on repère plein de choses. C'est une étude raisonnée de procédés grammaticaux, lexicaux, rhétoriques, énnonciatifs, etc... Dans le but de créer du sens. Le sens doit guider l'analyse. \par 

Privilégier le commentaire composé. Le commentaire linéaire est banni. Comment procéder?\par 

D'abord être attentif aux formes marquées (rupture de construction syntaxique, mais aussi rupture des traditions littéraires). Essayer de repérer les abandonces de formes particulière (abondance d'épitète derrière le nom par exemple), mais aussi les formes particulières comme les figures de style. Etre attentif aussi au siècle, le savoir littéraire comes in clutch. On a pas les mêmes attentes de La Fayette et de Beckett. \par

Notion d'horizon d'attente de Jauss, tout texte est fondé sur des attentes et une participation du lecteur. Pour analyser un texte, il faut donc connaitre les attentes du texte (dépendant du siècle, du genre, du moment dans l'ouvrage, public visé, etc). Pour comprendre la stylistique, il faut maitriser un dictionnaire de base de la littérature et de la linguistique et il faut avoir une compétence encyclopédique qui comprend des sénarios préfabriqués. \par 

Exemple: l'autobiographie. L'autobiographie est gérée par le pacte autobiographique défini par LeJeune? qui explique qu'il y a le narrant et le narré. \par 

Questionnaire à déployer face à une oeuvre. Le questionnaire dépend évidemment de l'horizon, du siècle et type d'oeuvre. Mais voici les questions de base, du macro au micro
\begin{enumerate}
    \item Quel est le sens du texte? Donner un titre au passage. Quels sont les enjeux? Quel est le stéréotypes littéraires? (ex: rencontre amoureuse, arrivée dans la gloire) Quel est le topoï
    \item Quel est le genre du texte? Quelles sont les contraintes associées, les attentes de l'ère et du genre. Quel est l'horizon d'attente? Est ce que le texte respecte cet horizon, ces contraintes? Il faut mobiliser les connaissances littéraires. 
    \item Quel est le type du texte? Jean Michel Albin: Une oeuvre est fondée sur une série de séquences. Il existe 5 types de séquence. La séquence narrative, la séquence descriptive, la séquence argumentative, la séquence explicative, et la séquence dialogale. 
    \item Quelle est l'énonciation? Qui parle, a qui, de quoi? Enonciation de récit: texte non ancré, enonciation historique. C'est quand le locuteur efface sa présence, comme si le récit se déroulait tout seul. Il n'y a aucune référence à la situation d'énonciation. On le reconnait à l'utilisation de la 3eme personne, du passé simple ou du présent de narration. Surtout, il n'y a pas de déictique: aucune marque de subjectivité. Remarquer la présence de subjectivité du locuteur, étude de la modalisation (attitude du locuteur face à un énoncé). On sent sa présence. Enonciation de discours. Mesurer le degré d'adhésion du locuteur a son énoncé.
    \item Quel est le registre? Un texte a toujours un ou plusieurs registres. Un monologue peut être élégiaque, même dans une comédie. Il peut y avoir du registre satirique aussi. C'est l'effet particulier produit sur le lecteur. Une même scène peut être racontée avec des registres différents. 
    \item Quelle est la progression textuelle? Qu'est ce qui fait que chaque phrase a un lien avec la suivante? La cohésion textuelle correspond aux elements grammaticaux et lexicaux pour créer une unité dans le texte. Il y a donc des marques de cohésion. La cohérence (ne pas confondre avec cohésion) en revanche et lui à la suite logique dans les idées entre les phrases. Fondé sur des règles de répétition et de progression, il y a un fil conducteur. 
    \item Stylistique de la phrase et/ou du vers
    \item Stylistique du mot. Relation sémantique entre les mots, champs dérivationel, champs sémantique, champs lexical, isotopie sémantique. Signifiant vs signifié
    \item Stylistique des figures
\end{enumerate}

\section{Méthode du commentaire stylistique} 
Introduction, Développement, Conclusion. Pas de plan dialectique, pas de 3eme partie nécessaire. Il y a une tendance 3eme partie méta/lecteur: souvent inutile \par

L'introduction se fait en plusieurs parties. D'abord, Situation du passage. On présente ses enjeux, on l'introduit, on le résume, où il se situe dans l'oeuvre. Puis on le caractérise, par un type et/ou un sous type. 3emement, la problématique. Doit présenter comment il se singularise. La question du genre et du type de texte permet une transition vers le projet de lecture. Annoncer le plan clairement. Avoir des sous-titre techniques.

\end{document}